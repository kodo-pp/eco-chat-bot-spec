\section{Требования к программе}
\label{sec:req}
\subsection{Требования к функциональности}
    \label{sec:req:fn}
    Чат-бот должен поддерживать следующие функции:
    \begin{enumerate}
        \item
            Доступ пользователей к внутренней базе знаний
        \item
            Редактирование базы знаний пользователями, наделёнными на это правом
        \item
            Рассылка новостей, связанных с экологической повесткой, подписавшимся на рассылку
            пользователям
        \item
            Автоматический сбор новостей с интернет-ресурсов для их дальнейшей рассылки
    \end{enumerate}

    \subsubsection{База знаний}
        \label{sec:req:fn:kb}
        Внутренняя база знаний представляет из себя иерархическую систему из разделов
        и заметок. Вне зависимости от наполнения базы знаний информацией, в ней существует единственный
        корневой раздел, не имеющий имени. Каждый раздел, не являющийся корневым, а также
        каждая заметка принадлежат тому или иному разделу (в том числе корневому). Циклическая
        принадлежность разделов (например, раздел \(A\) принадлежит разделу \(B\), который, в свою
        очередь, принадлежит разделу \(A\)) запрещена; также никакой раздел не может принадлежать самому
        себе. Корневой раздел не принадлежит никакому разделу. Таким образом, разделы, заметки и отношение
        принадлежности образуют иерархическую структуру --- дерево разделов. Эта структура напоминает
        дерево файлов и папок в файловой системе. Схематичное изображение примера дерева разделов
        показано на рис.~\ref{fig:req:fn:kb:tree}.
        \begin{figure}[h]
            \centering
            \begingroup
            \newlength{\treeindent}
            \newlength{\treeskip}
            \setlength{\treeindent}{2em}
            \setlength{\treeskip}{-3ex}
            \newcounter{treeline}
            \setcounter{treeline}0
            \newcommand{\mypoint}[2]{(#2 * \treeindent, #1 * \treeskip)}
            \newcommand{\outpoint}[2]{(#2 * \treeindent + 0.75em, #1 * \treeskip - 1.5ex)}
            \newcommand{\midpoint}[2]{(#2 * \treeindent + 0.75em, #1 * \treeskip)}
            \newcommand{\inpoint}[2]{(#2 * \treeindent - 0.1em, #1 * \treeskip)}
            \newcommand{\mynode}[2]{
                \node at \mypoint{\thetreeline}{#1} [anchor = west] {#2};%
                \stepcounter{treeline}%
            }
            \begin{tikzpicture}
                \mynode 0 {\emjfolder{} Корневой раздел}
                    \mynode 1 {\emjfolder{} Раздел <<Экология в Вышке>>}
                        \mynode 2 {\emjfolder{} Раздел <<Сбор отходов>>}
                            \mynode 3 {\emjnote{} Заметка <<Раздельный сбор мусора>>}
                            \mynode 3 {\emjnote{} Заметка <<Места сбора батареек>>}
                        \mynode 2 {\emjnote{} Заметка <<Планы развития>>}
                    \mynode 1 {\emjfolder{} Раздел <<О боте>>}
                        \mynode 2 {\emjnote{} Заметка <<Справка>>}
                    \mynode 1 {\emjnote{} Заметка <<Основная информация>>}
                \draw \outpoint00 -- \midpoint80;
                    \draw \midpoint10 -- \inpoint11;
                    \draw \midpoint60 -- \inpoint61;
                    \draw \midpoint80 -- \inpoint81;
                \draw \outpoint11 -- \midpoint51;
                    \draw \midpoint21 -- \inpoint22;
                    \draw \midpoint51 -- \inpoint52;
                \draw \outpoint22 -- \midpoint42;
                    \draw \midpoint32 -- \inpoint33;
                    \draw \midpoint42 -- \inpoint43;
                \draw \outpoint61 -- \midpoint71;
                    \draw \midpoint71 -- \inpoint72;
            \end{tikzpicture}
            \endgroup
            \caption{Пример дерева разделов.}
            \label{fig:req:fn:kb:tree}
        \end{figure}

        Каждый не-корневой раздел имеет имя. Каждая заметка имеет имя и содержимое.
        Именем явлется строка символов Юникода, поддерживаемых мессенджером, длиной
        не более 50~символов. У разделов и заметок, непосредственно принадлежащих одному и тому же
        разделу, имена должны различаться. Таким образом, запрещено наличие в одном разделе
        (а) двух разделов, (б) двух заметок и (в) раздела и заметки с одинаковыми именами.

        Содержимое каждой заметки является текстом длиной не более 3500~символов, а также может содержать
        не более 10~вложений. Каждое вложение --- это либо изображение разрешением не больше
        \(1920 \times 1080\)~пикселей, либо видео размером не более 50~МБ\footnote{
            Здесь и далее подразумеваются двоичные единицы измерения: \(1~\text{МБ} = 1024~\text{КБ}
            = 1048576~\text{Б}\).
        }, либо файл размером не более 50~МБ. Текст и вложения должны быть доступны пользователю
        через стандартные средства мессенджера. Допускается отправка текста и вложений
        отдельными соседними сообщениями.

        Навигация и просмотр базы знаний доступны всем пользователям.
        Пользователи, имеющие на это право, могут также редактировать базу знаний.
        Способ навигации по базе знаний, просмотра заметок и редактирования базы знаний
        описан в разделе \ref{sec:req:ui}.
        Право на редактирование описано в разделе
        \ref{sec:req:sec:privs}.

        Определённые разделы базы знаний могут быть виртуальными. Это означает, что их содержимое
        (в том числе содержимое вложенных в него разделов)
        генерируется автоматически, его нельзя создать, изменить или удалить вручную,
        а сам раздел нельзя удалить, переименовать или переместить.

        База знаний содержит один виртуальный раздел под названием <<Архив рассылок>>,
        находящийся в корневом разделе. В нём для каждой поддерживаемой рассылки
        (см.~раздел \ref{sec:req:fn:newsletter}) находится
        подраздел, в котором как заметки представлены все сообщения, которые бот присылал пользователям
        в рамках этой рассылки. Внутри раздела каждой рассылки допустимо использовать любую
        удобную структуру подразделов, однако она должна строиться по одинаковым принципам для всех рассылок.
        Разделы с рассылками, недоступными для получения определённому пользователю не должны быть ему
        видны или доступны, однако это не распространяется на рассылки, к которым пользователь имеет
        доступ, но на которые он не подписан.

        Не более 5 материалов (разделов или заметок) во всей базе знаний могут быть закреплены в
        главном меню одновременно.  Пользователи, имеющие право на редактирование базы знаний,
        могут закрепить какой-либо материал в главном меню при условии непревышения лимита на
        количество закреплённых материалов, а также открепить один или несколько материалов из
        главного меню. Порядок сортировки закреплённых материалов в главном меню определяется
        реализацией.

    \subsubsection{Рассылки}
        \label{sec:req:fn:newsletter}
        Бот должен производить рассылки информационных сообщений
        пользователям, подписанным на них. Список рассылок,
        их содержимое и частота отправки указана в последующих разделах документа.
        Если для какой-либо рассылки не указана частота её отправки, отправка сообщений происходит
        незамедлительно при появлении данных. Однако в таком случае задержка в отправке сообщений допускается,
        если она обусловлена технической необходимостью или правилами, описанными ниже в данном разделе.

        В рамках организации рассылок бот не должен присылать сообщения пользователям,
        которые в данный момент находятся в интерактивном состоянии (см.~раздел~\ref{sec:req:ui:states}),
        однако он может уведомить их о том, что те или иные рассылки для них готовы.
        После выхода пользователя из интерактивного состояния бот должен повторить попытку
        рассылки не ранее, чем через 30~секунд, но не позднее, чем через 30~минут.
        Если, руководствуясь данными правилами, бот не может отправить пользователю рассылку
        более 24~часов, то допускается не отправлять данную рассылку этому пользователю в этот раз.

        Также бот не должен присылать какие-либо сообщения пользователям в первоначальном
        состоянии (см.~раздел~\ref{sec:req:ui:states}) кроме непосредственного ответа на действия
        пользователя. Отправка рассылок пользователям в этом состоянии не осуществляется. Если
        во время рассылки пользователь находится в этом состоянии, то данную рассылку он не получает.

    \subsubsection{Сбор новостей}
        \label{sec:req:fn:grabnews}
        Не реже, чем один раз в день, бот совершает запросы на Интернет-ресурсы с целью получения с них
        новостей, связанных с экологической повесткой. Перечень собираемой информации:
        \begin{enumerate}
            \item
                Загловок новости
            \item
                Текст новости
            \item
                Прямая ссылка на источник новости (ресурс, с которого новость получена)
            \item
                Автор новости (если указан)
            \item
                Дата публикации (если указана)
        \end{enumerate}

        Данная операция происходит автоматически, без участия человека. При невозможности сбора данных
        с одного или нескольких ресурсов по той или иной причине бот посылает сервисное уведомление
        с описанием ошибки, а также записывает информацию об ошибке в системный журнал.

        При риске возникновения препятствий к автоматическому сбору новостей (например, CAPTCHA-проверок)
        или если правила пользования ресурса, с которого собирается информация, запрещает автоматизированный
        доступ к новостям, допускается брать новости из RSS-ленты соответствующего ресурса при её наличии.
        В таком случае, допускается не получать полный текст новости, сохраняя лишь заголовок,
        ссылку на источник и иную информацию при её наличии.

        Полученные новости сохраняются во внутреннюю базу данных бота и потом используются для новостной
        рассылки, описанной в разделе \ref{sec:req:fn:news}.

        \todo{Перечень ресурсов, с которых необходимо собирать новости.}

    \subsubsection{Новостная рассылка}
        \label{sec:req:fn:news}
        Один раз в сутки бот
        должен рассылать всем пользователям, подписанным на новостную рассылку, новости с внешних
        ресурсов, которые были собраны за этот день. Пользователю
        должна выводиться вся информация, сохранённая в базе данных, перечисленная в разделе
        \ref{sec:req:fn:grabnews}. Если текст новости длиннее 3500~символов или не помещается в
        стандартное сообщение мессенджера, допускается приводить только его начало, указывая,
        что по ссылке на источник откроется полный текст новости.

        Каждая новость присылается одним сообщением, не содержит вложений и
        должна быть маркирована хэштегом \hbox{\texttt{\#новость}}.

    \subsubsection{Рассылка сервисных уведомлений}
        \label{sec:req:fn:service}
        При возникновении определённых ситуаций, описанных в данном техническом задании,
        бот рассылает сервисные уведомления тем пользователям, которые на них подписаны и имеют
        право их получать. Сервисное уведомление представляет из себя одно сообщение с текстом
        не длиннее 3500~символов и не более, чем одним вложением.

        Каждое сервисное уведомление должно быть промаркировано хэштегом \hbox{\texttt{\#сервисное}}.

    \subsubsection{Обратная связь}
        \label{sec:req:fn:feedback}
        Любой пользователь имеет возможность оставить свои пожелания или замечания в специальном
        разделе бота. При отправке обратной связи необходимо выбрать тему (категорию) из списка,
        представленного ниже, и написать текстовое сообщение с обратной связью.
        Также допустимо приложить не более 10 вложений (согласно определению, данному в
        разделе~\ref{sec:req:fn:kb}). Полученные сообщения рассылаются в рамках рассылки,
        описанной в разделе~\ref{sec:req:fn:feedbacknl}.

        Перечень тем обратной связи:
        \begin{itemize}
            \item
                Зелёная вышка
            \item
                Бот
            \item
                Предложить экологическую инициативу
            \item
                Другое
        \end{itemize}

    \subsubsection{Рассылка обратной связи}
        \label{sec:req:fn:feedbacknl}
        При получении обратной связи от пользователей бот рассылает сообщения с ней пользователям,
        которые имеют право получать данную информацию и подписаны на данную рассылку.
        Каждое сообщение с обратной связью должно быть промаркировано хэштегом \hbox{\texttt{\#обратнаясвязь}},
        а также хэштегом, соответствующим теме обратной связи.

    \subsubsection{Управление подпиской на рассылки}
        \label{sec:req:fn:subscriptions}
        Любой пользователь имеет возможность подписаться, отписаться и проверить статус подписки на
        любые рассылки, доступные этому пользователю. Перечень рассылок, поддерживаемых ботом:
        \begin{enumerate}
            \item
                Новостная рассылка, описанная в разделе \ref{sec:req:fn:news}.
                Доступна всем пользователям, по умолчанию подписка неактивна.
            \item
                Рассылка сервисных уведомлений, описанная в разделе \ref{sec:req:fn:service}.
                Доступна только пользователям, наделённых правом получения сервисных уведомлений,
                подписка автоматически становится активна, когда пользователя в первый
                раз наделяют необходимыми правами.
            \item
                Рассылка обратной связи, описанная в разделе \ref{sec:req:fn:feedbacknl}.
                Доступна только пользователям, наделённых правом получения обратной связи,
                подписка автоматически становится активна, когда пользователя в первый
                раз наделяют необходимыми правами.
        \end{enumerate}

    \subsubsection{Управление правами пользователей}
        \label{seq:req:fn:roles}
        Пользователи, наделённые правами администратора, могут выдавать другим
        пользователям права и отзывать их. Также они могут редактировать свои права.
        Поддерживаемые права пользователей описаны в разделе \ref{sec:req:sec:privs}.
        После любого редактирования прав пользователей должен остаться хотя бы один
        пользователь с правами администратора. Попытка изменения прав, не соблюдающая
        это ограничение, заканчивается ошибкой и отсутствием изменений в правах.

\subsection{Требования к интерфейсу}
    \label{sec:req:ui}
    Интерфейс взаимодействия с ботом (отправка текста, изображений, видео и файлов)
    должен обеспечиваться стандартными средствами мессенджера. В любой момент времени работы
    бота пользователю доступны:
    \begin{enumerate}
        \item
            История взаимодействия с ботом: сообщения, отправленные ботом и пользователем
            друг другу
        \item
            Состояние, в котором на текущий момент находится взаимодействие с ботом, определяемое
            по последнему сообщению, отправленному ботом\footnote{В данном документе встречаются формулировки
            <<состояние пользователя>>, <<состояние бота>> и <<состояние взаимодействия
            (между пользоватем и ботом)>>. Данные формулировки являются эквивалентными и имеют
            один и тот же смысл. Формулировка <<состояние бота>> всегда подразумевает конкретного пользователя,
            взаимодействующего с ботом, поскольку состояние не является глобальным: взаимодействие одного
            пользователя с ботом не влияет на состояние взаимодействия другого пользователя с этим ботом.}
        \item
            Доступные методы дальнейшего взаимодействия с ботом: поле для ввода текста и вложений
            и/или кнопки или ссылки с вариантами взаимодействия, нажатие на которые приведёт к передаче
            команды боту, в зависимости от текущего состояния взаимодействия
    \end{enumerate}
    В каждом состоянии пользователю доступен определённый набор действий, каждое из которых
    приводит к ответу со стороны бота и переходу в другое или то же самое состояние.
    Информация, отправляемая и принимаемая ботом зависит от текущего состояния и определяется в разделе
    \ref{sec:req:ui:states}.

    \subsubsection{Состояния взаимодействия пользователя с ботом}
        \label{sec:req:ui:states}
        \begingroup
        \newcommand{\action}[1]{\textit{#1}}%
        \newcommand{\state}[1]{\uline{#1}}%
        В данном разделе \action{курсивом} указаны действия пользователя, а \state{подчёркнутым текстом}
        --- состояния взаимодействия.

        Пользователь, который до этого не взаимодействовал с ботом, а также пользователь, который
        до этого \action{остановил} бота и ещё не \action{запустил} его вновь, начинают
        своё взаимодействие с ним в \hyperref[itm:req:ui:states:init]{\state{первоначальном}} состоянии.
        Дальнейшее взаимодействие происходит согласно списку состояний, указанному ниже.
        Если для какого-то состояния какой-то вариант ввода со стороны пользователя не указан в данном
        списке, бот должен в ответ на него отправить сообщение, объясняющее, что данный
        ввод некорректен и остаться в том же состоянии.

        Интерактивным называется состояние, в котором пользователь не находится в процессе выполнения
        какого-либо действия с ботом. Для каждого из перечисленных ниже состояний может быть указано,
        считается ли оно интерактивным или нет. Если для какого-то состояния такого указания нет,
        оно считается интерактивным.
        \begin{enumerate}
            \item \label{itm:req:ui:states:init}
                \state{Первоначальное} состояние \\
                Состояние, в котором пользователь ещё не начал общение с ботом.
                Пользователю должна быть доступна кнопка с текстом <<Начать>>\footnote{
                Везде в этом документе, где указан текст элементов интерфейса, допускается
                использование другого текста с таким же смыслом, если это не повредит восприятию
                интерфейса пользователем.}, за исключением случаев, когда это технически невозможно.

                Любой ввод со стороны пользователя кроме отправки сообщений с одним или несколькими
                вложениями должен восприниматься ботом как \action{запуск} бота, при этом бот должен прислать
                сообщение с приветствием и кратким описанием своих функций, и перейти в состояние
                \hyperref[itm:req:ui:states:mainmenu]{\state{главного меню}}.
                Если же пользователь отправляет сообщение с одним или несколькими
                вложениями, бот должен отправить сообщение с объяснением, что он не готов принять вложения
                в данный момент, и что для того, чтобы начать работу с ботом, требуется \action{запустить}
                его. Состояние при этом не должно измениться.

                Данное состояние не считается интерактивным.

            \item \label{itm:req:ui:states:mainmenu}
                \state{Главное меню} \\
                Состояние, в котором пользователь оказывается непосредственно после \action{запуска}
                бота или после завершения какого-либо действия с ботом. Пользователю должны быть доступны
                следующие интерактивные кнопки или ссылки:
                \begin{itemize}
                    \item
                        По одной кнопке на каждый закреплённый раздел и каждую закреплённую заметку.
                        При \action{нажатии} на кнопку закреплённого раздела пользователь переходит
                        в состояние
                        \hyperref[itm:req:ui:states:navx]
                        {\state{навигации по базе знаний в \placeholder{выбранном разделе}}}.
                        При \action{нажатии} на кнопку закреплённой заметки пользователь переходит
                        в состояние
                        \hyperref[itm:req:ui:states:view-note]
                        {\state{просмотра \placeholder{выбранной заметки}}}.
                        Кнопки доступны и видимы всем пользователям.
                    \item
                        <<Все статьи>>
                        При \action{нажатии} на кнопку пользователь переходит в состояние
                        \hyperref[itm:req:ui:states:navx]
                        {\state{навигации по базе знаний в корневом разделе}}.
                        Кнопка доступна и видима всем пользователям.
                    \item
                        <<Архив рассылок>>
                        При \action{нажатии} на кнопку пользователь переходит в состояние
                        \hyperref[itm:req:ui:states:navx]
                        {\state{навигации по базе знаний в разделе <<Архив рассылок>>}}.
                        Кнопка доступна и видима всем пользователям.
                    \item
                        По одной кнопке для каждого ресурса, с которыми настроена интеграция.
                        При \action{нажатии} пользователь переходит в состояние
                        \hyperref[itm:req:ui:states:integrationx]
                        {\state{взаимодействия с \placeholder{выбранным ресурсом}}}.
                        Кнопка доступна и видима всем пользователям.
                    \item
                        <<Панель администратора>>.
                        При \action{нажатии} пользователь переходит в состояние
                        \hyperref[itm:req:ui:states:adminpanel]
                        {\state{работы с панелью администратора}}.
                        Кнопка доступна и видима пользователям, имеющим права администратора.
                    \item
                        <<Подписки на рассылки>>.
                        При \action{нажатии} пользователь переходит в состояние
                        \hyperref[itm:req:ui:states:subscriptions]
                        {\state{управления подписками}}.
                        Кнопка доступна и видима всем пользователям.
                    \item
                        <<Обратная связь>>.
                        При \action{нажатии} пользователь переходит в состояние
                        \hyperref[itm:req:ui:states:feedback]
                        {\state{обратной связи}}.
                        Кнопка доступна и видима всем пользователям.
                    \item
                        <<Предложить экологическую инициативу>>.
                        При \action{нажатии} пользователь переходит в состояние
                        \hyperref[itm:req:ui:states:feedbackx]
                        {\state{обратной связи с темой <<Предложить экологическую инициативу>>}}.
                        Кнопка доступна и видима всем пользователям.
                \end{itemize}
                Пример возможного расположения элементов пользовательского интерфейса показан на
                рис.~\ref{fig:sketch:mainmenu} и рис.~\ref{fig:sketch:mainmenu-for-admins}.

                Данное состояние не считается интерактивным.

            \item \label{itm:req:ui:states:navx}
                \state{Навигация по базе знаний в разделе \(X\)}

                Состояние, в котором пользователь находится при перемещении по разделам
                базы знаний. Раздел \(X\) --- это любой существующий раздел в базе знаний (включая корневой).
                Пользователю должен выводиться полный путь к разделу (от корневого). Пример
                вывода пути ко вложенному разделу показан на рис.~\ref{fig:sketch:kb-pagination}.

                В интерфейсе должны быть перечислены все разделы и заметки, содержащиеся в разделе
                \(X\). При необходимости следует использовать постраничный вывод этой информации
                (не более 10 материалов на странице), давая пользователю возможность переключаться
                между страницами вывода. В таком случае необходимо указывать, сколько всего материалов
                в данном разделе и какие по номеру материалы отображаются в данный момент.
                Также пользователю должны быть доступны следующие интерактивные кнопки или ссылки:
                \begin{itemize}
                    \item
                        <<Вверх>>.
                        При \action{нажатии} пользователь переходит в состояние
                        \hyperref[itm:req:ui:states:navx]
                        {\state{навигации по базе знаний в разделе \(Y\)}}, где \(Y\)
                        --- это родительский раздел \(X\) (т.е. раздел \(X\) принадлежит
                        разделу \(Y\)).
                        Кнопка доступна всем пользователям и видима, если текущий раздел
                        не является корневым.
                    \item
                        <<В главное меню>>.
                        При \action{нажатии} пользователь переходит в состояние
                        \hyperref[itm:req:ui:states:mainmenu]
                        {\state{главного меню}}.
                        Кнопка доступна и видима всем пользователям.
                    \item
                        <<Редактировать этот раздел>>.
                        При \action{нажатии} пользователь переходит в состояние
                        \hyperref[itm:req:ui:states:edit-section]
                        {\state{редактирования \placeholder{текущего раздела}}}.
                        Кнопка доступна и видима пользователям, имеющим право на редактирования базы знаний,
                        если текущий раздел не является виртуальным и не находится в виртуальном разделе.
                    \item
                        По одной кнопке на каждый раздел и каждую заметку, содержащиеся в разделе \(X\).
                        При \action{нажатии} на кнопку раздела пользователь переходит в состояние
                        \hyperref[itm:req:ui:states:navx]
                        {\state{навигации по базе знаний в \placeholder{выбранном разделе}}}.
                        При \action{нажатии} на кнопку заметки пользователь переходит в состояние
                        \hyperref[itm:req:ui:states:view-note]
                        {\state{просмотра \placeholder{выбранной заметки}}}.
                        Кнопки доступны и видимы всем пользователям.
                    \item
                        Прочие элементы интерфейса для организации постраничного вывода.
                        При \action{нажатии} пользователь переходит между страницами
                        вывода, не меняя состояния.
                        Кнопка доступна всем пользователям и видима, если полный
                        вывод не умещается на одну страницу.
                        Пример расположения таких элементов интерфейса показан
                        на рис.~\ref{fig:sketch:kb-pagination}.
                \end{itemize}
                Пример возможного расположения элементов пользовательского интерфейса показан на
                рис.~\ref{fig:sketch:kb-navigation-for-viewers} и
                рис.~\ref{fig:sketch:kb-navigation-for-editors}.

            \item \label{itm:req:ui:states:view-note}
                \state{Просмотр заметки \(X\)}

                Состояние, в котором пользователь имеет возможность прочитать текст и/или загрузить
                вложения определённой заметки. \(X\) может быть любой существующей заметкой.

                Бот отправляет пользователю одно или несколько сообщений (согласно правилам, указанным в
                данном техническом задании), в которых содержится следующая информация:
                \begin{itemize}
                    \item
                        Название заметки
                    \item
                        Текст заметки
                    \item
                        Все вложения, принадлежащие данной заметке (при их наличии)
                \end{itemize}

                Пользователю должны быть доступны следующие интерактивные кнопки или ссылки:
                \begin{itemize}
                    \item
                        <<Назад>>.
                        При \action{нажатии} пользователь переходит в состояние
                        \hyperref[itm:req:ui:states:navx]
                        {\state{навигации по базе знаний в разделе \(Y\)}}, где \(Y\)
                        --- это родительский раздел по отношению к заметке \(X\)
                        (т.е. заметка \(X\) принадлежит разделу \(Y\)).
                        Кнопка доступна и видима всем пользователям.
                    \item
                        <<В главное меню>>.
                        При \action{нажатии} пользователь переходит в состояние
                        \hyperref[itm:req:ui:states:mainmenu]
                        {\state{главного меню}}.
                        Кнопка доступна и видима всем пользователям.
                    \item
                        <<Редактировать>>.
                        При \action{нажатии} пользователь переходит в состояние
                        \hyperref[itm:req:ui:states:edit-note]
                        {\state{редактирования заметки}}.
                        Кнопка доступна и видима пользователям, имеющим право на редактирование базы знаний.
                    \item
                        <<Переименовать>>.
                        При \action{нажатии} пользователь переходит в состояние
                        \hyperref[itm:req:ui:states:rename-kbo]
                        {\state{переименования материала}}.
                        Кнопка доступна и видима пользователям, имеющим право на редактирование базы знаний.
                    \item
                        <<Переместить в другой раздел>>.
                        При \action{нажатии} пользователь переходит в состояние
                        \hyperref[itm:req:ui:states:move-kbo]
                        {\state{перемещения материала}}.
                        Кнопка доступна и видима пользователям, имеющим право на редактирование базы знаний.
                    \item
                        <<Удалить>>.
                        При \action{нажатии} пользователь переходит в состояние
                        \hyperref[itm:req:ui:states:delete-kbo]
                        {\state{подтверждения удаления материала}}.
                        Кнопка доступна и видима пользователям, имеющим право на редактирование базы знаний.
                    \item
                        <<Закрепить в главном меню>> или <<Открепить из главного меню>>.
                        При \action{нажатии} заметка становится закреплённой в главном меню или теряет этот
                        статус. Если при нажатии кнопки <<Закрепить в главном меню>> лимит на количество
                        закреплённых в главном меню материалов превышается, бот должен сообщить об
                        этом и не изменять статус заметки. В любом случае, состояние пользователя
                        не изменяется.
                        Кнопка доступна и видима пользователям, имеющим право на редактирование базы знаний.
                \end{itemize}

                Пример возможного расположения элементов пользовательского интерфейса показан на
                рис.~\ref{fig:sketch:kb-note-for-viewers} и
                рис.~\ref{fig:sketch:kb-note-for-editors}.

            \item \label{itm:req:ui:states:integrationx}
                \state{Взаимодействие с ресурсом \(X\)}

                Данный класс состояний описан в разделе \ref{par:req:ui:states:integrations}.

            \item \label{itm:req:ui:states:subscriptions}
                \state{Управление подписками}

                В данном состоянии пользователь видит список доступных ему рассылок и подписан ли
                он на каждую из них, а также может подписаться на какие-либо рассылки или отписаться
                от них.

                Для каждой доступной пользователю подписки должны выводиться
                следующие данные:
                \begin{itemize}
                    \item
                        Название рассылки
                    \item
                        Краткое описание рассылки
                    \item
                        Статус подписки пользователя на эту рассылку (подписан или не подписан)
                    \item
                        Кнопка или ссылка для подписки или отписки от данной рассылки
                \end{itemize}

                При \action{нажатии} на кнопку или ссылку для подписки на рассылку или отписки от неё
                бот должен прислать пользователю уведомление о соответствующем изменении в статусе
                его подписки.

                Также пользователю должна быть доступна кнопка или ссылка <<В главное меню>>.
                При \action{нажатии} пользователь переходит в состояние
                \hyperref[itm:req:ui:states:mainmenu]
                {\state{главного меню}}.

                Пример возможного расположения элементов пользовательского интерфейса показан на
                рис.~\ref{fig:sketch:subscriptions}.

            \item \label{itm:req:ui:states:adminpanel}
                \state{Работа с панелью администратора}

                В данном состоянии пользователю показывается меню действий, доступных администраторам.
                Пользователю должны быть доступны следующие интерактивные кнопки или ссылки:
                \begin{itemize}
                    \item
                        <<В главное меню>>.
                        При \action{нажатии} пользователь переходит в состояние
                        \hyperref[itm:req:ui:states:mainmenu]
                        {\state{главного меню}}.
                    \item
                        <<Права пользователей>>.
                        При \action{нажатии} пользователь переходит в состояние
                        \hyperref[itm:req:ui:states:user-privs]
                        {\state{управления правами пользователей}}.
                \end{itemize}

                Все кнопки доступны и видимы только администраторам.

                Пример возможного расположения элементов пользовательского интерфейса показан на
                рис.~\ref{fig:sketch:adminpanel}.

            \item \label{itm:req:ui:states:user-privs}
                \state{Управление правами пользователей}

                В данном состоянии пользователю показывается список всех пользователей
                имеющих специальные права (см. раздел~\ref{sec:req:sec:privs}), список прав каждого
                упомянутого пользователя, а также предоставляется
                возможность выдать пользователям права или отозвать их.
                Пользователю должны быть доступны следующие интерактивные кнопки или ссылки:
                \begin{itemize}
                    \item
                        По одной кнопке или ссылке для управления правами каждого пользователя.
                        При \action{нажатии} пользователь переходит в состояние
                        \hyperref[itm:req:ui:states:user-privsx]
                        {\state{управлению правами \placeholder{выбранного пользователя}}}.
                    \item
                        <<Выдать права другому пользователю>>.
                        При \action{нажатии} пользователь переходит в состояние
                        \hyperref[itm:req:ui:states:user-privs-unlisted]
                        {\state{управлению правами пользователя не из списка}}.
                    \item
                        <<В главное меню>>.
                        При \action{нажатии} пользователь переходит в состояние
                        \hyperref[itm:req:ui:states:mainmenu]
                        {\state{главного меню}}.
                    \item
                        <<В панель администратора>>.
                        При \action{нажатии} пользователь переходит в состояние
                        \hyperref[itm:req:ui:states:adminpanel]
                        {\state{панели администратора}}.
                \end{itemize}

                Все кнопки доступны и видимы только администраторам.

                Пример возможного расположения элементов пользовательского интерфейса показан на
                рис.~\ref{fig:sketch:privs}.

            \item \label{itm:req:ui:states:user-privsx}
                \state{Управление правами пользователя \(X\)}

                В данном состоянии пользователю показывается список прав, которые имеет пользователь \(X\)
                и предоставляется возможность выдать ему права или отозвать их.
                Пользователю должны быть доступны следующие интерактивные кнопки или ссылки:
                \begin{itemize}
                    \item
                        <<Запретить/Разрешить редактирование базы знаний>>
                        (в зависимости от текущего набора прав).
                        При \action{нажатии} отзывает или выдаёт пользователю \(X\) право на редактирование
                        базы знаний. Состояние не меняется.
                    \item
                        <<Запретить/Разрешить получение сервисных уведомлений>>
                        (в зависимости от текущего набора прав).
                        При \action{нажатии} отзывает или выдаёт пользователю \(X\) право на получение
                        сервисных уведомлений. Состояние не меняется.
                    \item
                        <<Запретить/Разрешить получение обратной связи>>
                        (в зависимости от текущего набора прав).
                        При \action{нажатии} отзывает или выдаёт пользователю \(X\) право на получение
                        обратной связи. Состояние не меняется.
                    \item
                        <<Сделать администратором / Отозвать статус администратора>>
                        (в зависимости от текущего набора прав).
                        При \action{нажатии} отзывает или выдаёт пользователю \(X\) статус администратора.
                        Состояние не меняется. Если статус отозвать невозможно из-за нарушения ограничений,
                        указанных в этом документе, бот сообщает пользователю об ошибке и не изменяет
                        права пользователя \(X\).
                    \item
                        <<Назад>>.
                        При \action{нажатии} пользователь переходит в состояние
                        \hyperref[itm:req:ui:states:user-privs]
                        {\state{управления правами пользователей}}.
                \end{itemize}

                Все кнопки доступны и видимы только администраторам.

                Пример возможного расположения элементов пользовательского интерфейса показан на
                рис.~\ref{fig:sketch:privs}.

            \item \label{itm:req:ui:states:user-privs-unlisted}
                \state{Управление правами пользователя не из списка}

                В данном состоянии можно управлять правами пользователя, не указанного в списке,
                описанном в состоянии
                \hyperref[itm:req:ui:states:user-privs]
                {\state{управления правами пользователей}}.
                Для этого пользователь должен отправить боту сообщение с указанием пользователя,
                к которому нужно применить изменения.
                Пользователю должны быть доступны следующие интерактивные кнопки или ссылки:
                \begin{itemize}
                    \item
                        <<Назад>>.
                        При \action{нажатии} пользователь переходит в состояние
                        \hyperref[itm:req:ui:states:user-privs]
                        {\state{управления правами пользователей}}.
                \end{itemize}

                Все кнопки доступны и видимы только администраторам.

                Если перед этим не были совершены другие действия, то сообщение от пользователя
                в одном из форматов, указанных ниже, должно быть принято в качестве идентификационной
                информации другого пользователя. Если пользователь не может быть найден по такой
                идентификационной информации, бот выдаёт сообщение об ошибке и не меняет состояние.
                В случае успеха пользователь переходит в состояние
                \hyperref[itm:req:ui:states:user-privsx]
                {\state{управления правами \placeholder{указанного пользователя}}}.

                Возможные форматы сообщения с идентификационной информацией:
                \begin{itemize}
                    \item
                        \texttt{\emph{username}}
                    \item
                        \texttt{@\emph{username}}
                    \item
                        \texttt{id:\emph{code}}
                \end{itemize}
                Здесь \texttt{\emph{username}} --- это строка с именем пользователя, а
                \texttt{\emph{code}} --- это числовой идентификатор пользователя. Подробнее
                об идентификации пользователей написано в разделе~\ref{sec:req:sec:id}.

                Пример возможного расположения элементов пользовательского интерфейса показан на
                рис.~\ref{fig:sketch:privs}.

            \item \label{itm:req:ui:states:feedback}
                \state{Обратная связь}

                В данном состоянии пользователь может выбрать тему обратной связи.
                Пользователю должны быть доступны следующие интерактивные кнопки или ссылки:
                \begin{itemize}
                    \item
                        <<В главное меню>>.
                        При \action{нажатии} пользователь переходит в состояние
                        \hyperref[itm:req:ui:states:mainmenu]
                        {\state{главного меню}}.
                        Кнопка доступна и видима всем пользователям.
                    \item
                        По одной кнопке для каждой доступной темы обратной связи
                        (см. раздел~\ref{sec:req:fn:feedback}).
                        При \action{нажатии} пользователь переходит в состояние
                        \hyperref[itm:req:ui:states:feedbackx]
                        {\state{обратной связи с \placeholder{выбранной темой}}}.
                        Кнопка доступна и видима всем пользователям.
                \end{itemize}

                Пример возможного расположения элементов пользовательского интерфейса показан на
                рис.~\ref{fig:sketch:feedback}.

            \item \label{itm:req:ui:states:feedbackx}
                \state{Обратная связь с темой \(X\)}

                В данном состоянии пользователь может написать сообщение с обратной связью
                с выбранной темой. \(X\) может быть любой доступной темой обратной связи
                (см. раздел~\ref{sec:req:fn:feedback}).
                Пользователю должны быть доступны следующие интерактивные кнопки или ссылки:
                \begin{itemize}
                    \item
                        <<В главное меню>>.
                        При \action{нажатии} пользователь переходит в состояние
                        \hyperref[itm:req:ui:states:mainmenu]
                        {\state{главного меню}}.
                        Кнопка доступна и видима всем пользователям.
                \end{itemize}

                Если перед этим не были совершены другие действия, то любое сообщение от пользователя,
                возможно, включающее вложения, должно быть принято в качестве содержимого обратной связи,
                если оно не нарушает установленные в этом документе лимиты. В случае успеха
                пользователь переходит в состояние
                \hyperref[itm:req:ui:states:mainmenu]
                {\state{главного меню}}.

                Пример возможного расположения элементов пользовательского интерфейса показан на
                рис.~\ref{fig:sketch:feedback}.

            \item \label{itm:req:ui:states:edit-note}
                \state{Редактирование заметки}

                В данном состоянии пользователь может отправить сообщение с новым содержимым
                заметки, а также указать, что делать с существующими вложениями.
                Пользователю должны быть доступны следующие интерактивные кнопки или ссылки:
                \begin{itemize}
                    \item
                        <<Сохранить существующие вложения>> или <<Не сохранять существующие вложения>>.
                        При \action{нажатии} переключается поведение существующих вложений: при настройке
                        <<сохранить>> они будут сохранены, а новые вложения (при наличии)
                        --- добавлены к ним (при условии соблюдения лимита на количество вложений);
                        при настройке <<не сохранять>> существующие вложения будут удалены, а новые
                        вложения (при наличии) их заменят.
                        В любом случае, состояние пользователя не изменяется.
                        Кнопка доступна и видима всем пользователям. Текст варьируется в зависимости от
                        текущей настройки: <<Сохранить существующие вложения>> показывается, когда настройка
                        выставлена в положение <<не сохранять>>, и наоборот.
                    \item
                        <<Назад>>.
                        При \action{нажатии} пользователь переходит в состояние
                        \hyperref[itm:req:ui:states:view-note]
                        {\state{просмотра \placeholder{текущей заметки}}}.
                        Кнопка доступна и видима всем пользователям.
                \end{itemize}

                Если перед этим не были совершены другие действия, то любое сообщение от пользователя,
                возможно, включающее вложения, должно быть принято в качестве нового содержимого заметки,
                если оно не нарушает установленные в этом документе лимиты. В случае успеха
                пользователь переходит в состояние
                \hyperref[itm:req:ui:states:view-note]
                {\state{просмотра \placeholder{текущей заметки}}}.

                Пример возможного расположения элементов пользовательского интерфейса показан на
                рис.~\ref{fig:sketch:edit-note}.

            \item \label{itm:req:ui:states:edit-section}
                \state{Редактирование раздела}

                В данном состоянии пользователь может изменить имя, расположение и содержимое не-виртуального
                раздела.
                Пользователю должны быть доступны следующие интерактивные кнопки или ссылки:
                \begin{itemize}
                    \item
                        <<Создать заметку>>.
                        При \action{нажатии} пользователь переходит в состояние
                        \hyperref[itm:req:ui:states:create-note]
                        {\state{создания заметки}}.
                    \item
                        <<Создать подраздел>>.
                        При \action{нажатии} пользователь переходит в состояние
                        \hyperref[itm:req:ui:states:create-section]
                        {\state{создания раздела}}.
                    \item
                        <<Переименовать>>.
                        При \action{нажатии} пользователь переходит в состояние
                        \hyperref[itm:req:ui:states:rename-kbo]
                        {\state{переименования материала}}.
                    \item
                        <<Переместить в другой раздел>>.
                        При \action{нажатии} пользователь переходит в состояние
                        \hyperref[itm:req:ui:states:move-kbo]
                        {\state{перемещения материала}}.
                    \item
                        <<Удалить>>.
                        При \action{нажатии} пользователь переходит в состояние
                        \hyperref[itm:req:ui:states:delete-kbo]
                        {\state{удаления материала}}.
                    \item
                        <<Закрепить в главном меню>> или <<Открепить из главного меню>>.
                        При \action{нажатии} раздел становится закреплённым в главном меню или теряет этот
                        статус. Если при нажатии кнопки <<Закрепить в главном меню>> лимит на количество
                        закреплённых в главном меню материалов превышается, бот должен сообщить об
                        этом и не изменять статус раздела. В любом случае, состояние пользователя
                        не изменяется.
                    \item
                        <<Назад>>.
                        При \action{нажатии} пользователь переходит в состояние
                        \hyperref[itm:req:ui:states:navx]
                        {\state{навигации по базе знаний в \placeholder{текущем разделе}}}.
                        Кнопка доступна и видима всем пользователям.
                \end{itemize}

                Все кнопки доступны и видимы только пользователям, имеющим право на редактирование
                базы знаний.

                Пример возможного расположения элементов пользовательского интерфейса показан на
                рис.~\ref{fig:sketch:edit-section}.

            \item \label{itm:req:ui:states:rename-kbo}
                \state{Переименование материала}

                В данном состоянии пользователь может отправить сообщение с новым именем для
                текущей заметки или текущего раздела, чтобы переименовать эту заметку или этот раздел.
                Пользователю должны быть доступны следующие интерактивные кнопки или ссылки:
                \begin{itemize}
                    \item
                        <<Назад>>.
                        При \action{нажатии} пользователь переходит в состояние
                        \hyperref[itm:req:ui:states:navx]
                        {\state{навигации по базе знаний в \placeholder{текущем разделе}}}.
                        Кнопка доступна и видима всем пользователям.
                \end{itemize}
                Если перед этим не были совершены другие действия, то сообщение от пользователя
                с новым названием раздела или заметки без вложений должно быть принято,
                и раздел или заметка должны быть переименованы соответствующим образом,
                если данное имя допустимо.
                В случае успеха пользователь переходит в состояние
                \hyperref[itm:req:ui:states:navx]
                {\state{навигации по базе знаний в \placeholder{текущем разделе}}}.

                Все кнопки и действия доступны и видимы только пользователям, имеющим право на редактирование
                базы знаний.

            \item \label{itm:req:ui:states:move-kbo}
                \state{Перемещение материала (текущий раздел \(X\))}

                Состояние, в котором пользователь выбирает раздел, в который следует переместить
                выбранную заметку или раздел, и сейчас находится в разделе \(X\).
                Раздел \(X\) --- это любой существующий раздел в базе знаний (включая корневой).
                Пользователю должен выводиться полный путь к разделу (от корневого). Пример
                вывода пути ко вложенному разделу показан на рис.~\ref{fig:sketch:kb-pagination}.
                При соблюении остальных требований данного пункта, интерфейс в данном состоянии
                должен быть как можно более похожим стилистически и визуально на интерфейс
                в состоянии
                \hyperref[itm:req:ui:states:navx]
                {\state{навигации по базе знаний в разделе \(X\)}}.

                В интерфейсе должны быть перечислены все разделы, содержащиеся в разделе
                \(X\). При необходимости следует использовать постраничный вывод этой информации
                (не более 10 элементов на странице), давая пользователю возможность переключаться
                между страницами вывода. В таком случае необходимо указывать, сколько всего подразделов
                в данном разделе и какие по номеру подраздел отображаются в данный момент.
                Также пользователю должны быть доступны следующие интерактивные кнопки или ссылки:
                \begin{itemize}
                    \item
                        <<Вверх>>, по одной кнопке на каждый подраздел, а также
                        элементы интерфейса для организации потраничного вывода при необходимости.
                        Кнопки работают аналогично соответствующим кнопкам из состояния 
                        \hyperref[itm:req:ui:states:navx]
                        {\state{навигации по базе знаний в разделе \(X\)}}, за исключением того,
                        что вместо состояния
                        \hyperref[itm:req:ui:states:navx]
                        {\state{навигации по базе знаний в разделе \(Y\)}}.
                        текущее состояние меняется на состояние
                        \hyperref[itm:req:ui:states:move-kbo]
                        {\state{перемещения материала (текущий раздел \(Y\))}}.
                    \item
                        <<Отменить перемещение>>.
                        При \action{нажатии} процесс перемещения материала отменяется.
                        Пользователь возвращается в состояние
                        \hyperref[itm:req:ui:states:navx]
                        {\state{навигации по базе знаний в разделе \(X\)}}.
                    \item
                        <<Переместить сюда>>.
                        При \action{нажатии} бот перемещает выбранный материал в выбранный раздел,
                        и сообщает пользователю о результате операции. В любом случае, пользователь
                        переходит в состояние
                        \hyperref[itm:req:ui:states:navx]
                        {\state{навигации по базе знаний в разделе \(N\)}}.
                        В случае успешного перемещения раздел \(N\) --- это перемещённый раздел
                        в новом месте. В случае неуспешного перемещения раздел \(N\) --- это раздел \(X\)
                        без каких-либо изменений.
                \end{itemize}

                Все кнопки доступны и видимы только пользователям, имеющим право на редактирование базы
                знаний.

                Пример возможного расположения элементов пользовательского интерфейса показан на
                рис.~\ref{fig:sketch:move-note}.


            \item \label{itm:req:ui:states:delete-kbo}
                \state{Подтверждение удаления материала}

                В данном состоянии пользователь подтверждает или отменяет удаление выбранного
                материала. Пользователю обязательно должен отображаться полный путь к материалу
                на удаление.
                \begin{itemize}
                    \item
                        <<Удалить>>.
                        При \action{нажатии} выбранный материал (и все вложенные в него материалы,
                        если это раздел) удаляются из базы знаний без штатной возможности восстановления.
                        При возникновении ошибки бот должен вывести её пользователю и немеденно прервать
                        операцию удаления.
                        При успешном удалении пользователь переходит в состояние
                        \hyperref[itm:req:ui:states:navx]
                        {\state{навигации по базе знаний в разделе \(Y\)}},
                        где \(Y\) --- это раздел, в котором непосредственно содержался удалённый раздел.
                        При ошибке удаления пользователь переходит в состояние
                        \hyperref[itm:req:ui:states:navx]
                        {\state{навигации по базе знаний в \placeholder{текущем разделе}}}.
                    \item
                        <<Не удалять>>.
                        При \action{нажатии} операция удалений материала отменяется.
                        Пользователь переходит в состояние
                        \hyperref[itm:req:ui:states:navx]
                        {\state{навигации по базе знаний в \placeholder{текущем разделе}}}.
                \end{itemize}

                Пример возможного расположения элементов пользовательского интерфейса показан на
                рис.~\ref{fig:sketch:delete-note} и рис.~\ref{fig:sketch:delete-section}.

            \item \label{itm:req:ui:states:create-note}
                \state{Создание заметки}

                В данном состоянии пользователь начинает процесс создания новой заметки и может
                отправить сообщение с именем создаваемой заметки.

                Также пользователю должны быть доступны следующие интерактивные кнопки или ссылки:
                \begin{itemize}
                    \item
                        <<Назад>>.
                        При \action{нажатии} пользователь отменяет создание заметки и
                        переходит в состояние
                        \hyperref[itm:req:ui:states:navx]
                        {\state{навигации по базе знаний в \placeholder{текущем разделе}}}.
                \end{itemize}
                
                Если перед этим не были совершены другие действия, то сообщение от пользователя
                с названием новой заметки без вложений должно быть принято ботом, и
                состояние взаимодействия должно измениться на
                \hyperref[itm:req:ui:states:create-notex]
                {\state{создание заметки с названием \placeholder{из этого сообщения}}}.

                Все кнопки и действия доступны и видимы только пользователям, имеющим право на редактирование
                базы знаний.

                Пример возможного расположения элементов пользовательского интерфейса показан на
                рис.~\ref{fig:sketch:add-note}.

            \item \label{itm:req:ui:states:create-notex}
                \state{Создание заметки с названием \(X\)}

                В данном состоянии пользователь продолжает процесс создания новой заметки и может
                отправить сообщение с содержанием заметки.

                Также пользователю должны быть доступны следующие интерактивные кнопки или ссылки:
                \begin{itemize}
                    \item
                        <<Назад>>.
                        При \action{нажатии} пользователь возвращается к шагу выбора имени для заметки
                        в состояние
                        \hyperref[itm:req:ui:states:create-note]
                        {\state{создания заметки}}.
                \end{itemize}
                
                Если перед этим не были совершены другие действия, то сообщение от пользователя
                с должно быть принято ботом, и новая заметка должна сохраниться.
                Бот должен сообщить пользователю об успехе или ошибке операции, и
                состояние взаимодействия должно измениться на состояние
                \hyperref[itm:req:ui:states:navx]
                {\state{навигации по базе знаний в \placeholder{текущем разделе}}}.

                Все кнопки и действия доступны и видимы только пользователям, имеющим право на редактирование
                базы знаний.

                Пример возможного расположения элементов пользовательского интерфейса показан на
                рис.~\ref{fig:sketch:add-note}.

            \item \label{itm:req:ui:states:create-section}
                \state{Создание раздела}

                В данном состоянии пользователь может отправить сообщение с именем раздела и таким образом
                создать пустой раздел с выбранным именем.

                Также пользователю должны быть доступны следующие интерактивные кнопки или ссылки:
                \begin{itemize}
                    \item
                        <<Назад>>.
                        При \action{нажатии} пользователь отменяет создание раздела и
                        переходит в состояние
                        \hyperref[itm:req:ui:states:navx]
                        {\state{навигации по базе знаний в \placeholder{текущем разделе}}}.
                \end{itemize}
                
                Если перед этим не были совершены другие действия, то сообщение от пользователя
                с названием новой заметки без вложений должно быть принято ботом, и
                бот должен создать в текущем разделе пустой подраздел с таким именем.
                Об успехе или ошибке операции необходимо сообщить пользователю.
                После выполнения операции состояние должно измениться на
                состояние \hyperref[itm:req:ui:states:navx]
                {\state{навигации по базе знаний в \placeholder{текущем разделе}}}.

                Все кнопки и действия доступны и видимы только пользователям, имеющим право на редактирование
                базы знаний.

                Пример возможного расположения элементов пользовательского интерфейса показан на
                рис.~\ref{fig:sketch:add-section}.
        \end{enumerate}

        При слишком долгом нахождении в каком-либо интерактивном состоянии без какой-либо активности
        со стороны пользователя, взаимодействие с ним автоматически переходит в состояние
        \hyperref[itm:req:ui:states:mainmenu]
        {\state{главного меню}}, при этом отменяются все начатые, но не завершённые операции.
        Время, которое должно пройти с момента последней активности пользователя до
        автоматического перехода в главное меню называется временем ожидания. Во всех интерактивных
        состояниях время ожидания составляет один час (3600~секунд), со следующими исключениями:
        \begin{itemize}
            \item
                В состоянии
                \hyperref[itm:req:ui:states:edit-note]
                {\state{редактирования заметки}}
                время ожидания состовляет 6~часов
            \item
                В состоянии
                \hyperref[itm:req:ui:states:create-note]
                {\state{создания заметки с именем \(X\)}}
                время ожидания состовляет 6~часов
        \end{itemize}

        \paragraph{Интеграция с внешними ресурсами}
            \label{par:req:ui:states:integrations}
            ~\par
            В данном разделе описывается пользовательский интерфейс для интеграции с внешними
            ресурсами и сопутствующие состояния взаимодействия с ботом.
            Интерфейс для интеграции с каждым ресурсом описан в своём подразделе.
            Каждый подраздел содержит описание состояний взаимодействия пользователя с ботом,
            связанных с одним ресурсом.

            \subparagraph{Зелёная вышка}
                \begin{itemize}
                    \item
                        \todo{\state{Взаимодействие с ресурсом <<Зелёная вышка>>}}
                \end{itemize}

            \subparagraph{Качество воздуха в Москве}
                \begin{itemize}
                    \item
                        \todo{\state{Взаимодействие с ресурсом <<Индекс качества воздуха в Москве>>}}
                \end{itemize}

            \subparagraph{Календарь событий}
                \begin{itemize}
                    \item
                        \todo{\state{Взаимодействие с ресурсом <<Календарь событий>>}}
                \end{itemize}
        \endgroup

\subsection{Требования к безопасности}
    \label{sec:req:sec}

    \subsubsection{Специальные права пользователей}
        \label{sec:req:sec:privs}
        \todo{Описание прав пользователей.}

    \subsubsection{Идентификация пользователей}
        \label{sec:req:sec:id}
        \todo{Описание способов идентификации пользователей.}

\subsection{Требования к надёжности}
    \label{sec:req:reliab}
    \todo{Требования к обеспечению надёжного функционирования,
    контроль входных и выходных данных, время восстановления после отказа, etc.}

\subsection{Условия эксплуатации}
    \label{sec:req:maint}
    \todo{Физические условия, вид обслуживания, количество и квалификация персонала для обслуживания}

\subsection{Требования к составу и параметрам технических средств}
    \label{sec:req:hw}
    \todo{Требования к серверу (аппаратные и программные) и к клиентским приложениям для мессенджера}

\subsection{Требования к информационной и программной совместимости}
    \label{sec:req:compat}
    \todo{Формат данных, методы решения, языки программирования, библиотеки, etc.}

\subsection{Требования к маркировке и упаковке}
    \label{sec:req:ship}
    \todo{Поставка программы заказчику, взаимодействие пользователей с программой}

\subsection{Требования к транспортированию и хранению}
    \label{sec:req:data}
    \todo{Условия хранения и передачи данных}

\subsection{Специальные требования}
    \label{sec:req:etc}
    \todo{Другие требования при наличии}


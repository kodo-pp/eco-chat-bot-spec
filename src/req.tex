\section{Требования к программе}
\label{sec:req}
\subsection{Требования к функциональности}
    \label{sec:req:fn}
    Чат-бот должен поддерживать следующие функции:
    \begin{enumerate}
        \item
            Доступ пользователей к внутренней базе знаний
        \item
            Редактирование базы знаний пользователями, наделёнными на это правом
        \item
            Рассылка новостей, связанных с экологической повесткой, подписавшимся на рассылку
            пользователям
        \item
            Автоматический сбор новостей с интернет-ресурсов для их дальнейшей рассылки
    \end{enumerate}

    \subsubsection{База знаний}
        \label{sec:req:fn:kb}
        Внутренняя база знаний представляет из себя иерархическую систему из разделов
        и заметок. Вне зависимости от наполнения базы знаний информацией, в ней существует единственный
        корневой раздел, не имеющий имени. Каждый раздел, не являющийся корневым, а также
        каждая заметка принадлежат тому или иному разделу (в том числе корневому). Циклическая
        принадлежность разделов (например, раздел \(A\) принадлежит разделу \(B\), который, в свою
        очередь, принадлежит разделу \(A\)) запрещена; также никакой раздел не может принадлежать самому
        себе. Корневой раздел не принадлежит никакому разделу. Таким образом, разделы, заметки и отношение
        принадлежности образуют иерархическую структуру --- дерево разделов. Эта структура напоминает
        дерево файлов и папок в файловой системе. Схематичное изображение примера дерева разделов
        показано на рис.~\ref{fig:req:fn:kb:tree}.
        \begin{figure}[h]
            \centering
            \begingroup
            \newcommand{\name}{\rmfamily\,}
            \begin{tabular}{>{\ttfamily}l}
                \name Корневой раздел \\
                +---\name Раздел «Экология в Вышке» \\
                |~~~+---\name Раздел «Переработка отходов» \\
                |~~~|~~~+---\name Заметка «Раздельный сбор мусора» \\
                |~~~|~~~+---\name Заметка «Места сбора батареек» \\
                |~~~+---\name Заметка «Планы развития» \\
                +---\name Раздел «О боте» \\
                |~~~+---\name Заметка «Справка» \\
                +---\name Заметка «Основная информация» \\
            \end{tabular}
            \endgroup
            \caption{Пример дерева разделов.}
            \label{fig:req:fn:kb:tree}
        \end{figure}

        Каждый не-корневой раздел имеет имя. Каждая заметка имеет имя и содержимое.
        Именем явлется строка символов Юникода, поддерживаемых мессенджером, длиной
        не более 300~символов. У разделов и заметок, непосредственно принадлежащих одному и тому же
        разделу, имена должны различаться. Таким образом, запрещено наличие в одном разделе
        (а) двух разделов, (б) двух заметок и (в) раздела и заметки с одинаковыми именами.

        Содержимое каждой заметки является текстом длиной не более 3500~символов, а также может содержать
        не более 10~вложений. Каждое вложение --- это либо изображение разрешением не больше
        \(1920 \times 1080\)~пикселей, либо видео размером не более 50~МБ\footnote{
            Здесь и далее подразумеваются двоичные единицы измерения: \(1~\text{МБ} = 1024~\text{КБ}
            = 1048576~\text{Б}\).
        }, либо файл размером не более 50~МБ. Текст и вложения должны быть доступны пользователю
        через стандартные средства мессенджера. Допускается отправка текста и вложений
        отдельными соседними сообщениями.

        Навигация и просмотр базы знаний доступны всем пользователям.
        Пользователи, имеющие на это право, могут также редактировать базу знаний.
        Способ навигации по базе знаний, просмотра заметок и редактирования базы знаний
        описан в разделе \ref{sec:req:ui}.
        Правила определения пользователей, имеющих право на редактирование, описаны в разделе
        \ref{sec:req:sec}.

    \subsubsection{Сбор новостей}
        \label{sec:req:fn:grabnews}
        Не реже, чем один раз в день, бот совершает запросы на Интернет-ресурсы с целью получения с них
        новостей, связанных с экологической повесткой. Перечень собираемой информации:
        \begin{enumerate}
            \item
                Загловок новости
            \item
                Текст новости
            \item
                Прямая ссылка на источник новости (ресурс, с которого новость получена)
            \item
                Автор новости (если указан)
            \item
                Дата публикации (если указана)
        \end{enumerate}

        Данная операция происходит автоматически, без участия человека. При невозможности сбора данных
        с одного или нескольких ресурсов по той или иной причине бот посылает сервисное уведомление
        с описанием ошибки, а также записывает информацию об ошибке в системный журнал.

        При риске возникновения препятствий к автоматическому сбору новостей (например, CAPTCHA-проверок)
        или если правила пользования ресурса, с которого собирается информация, запрещает автоматизированный
        доступ к новостям, допускается брать новости из RSS-ленты соответствующего ресурса при её наличии.
        В таком случае, допускается не получать полный текст новости, сохраняя лишь заголовок,
        ссылку на источник и иную информацию при её наличии.

        Полученные новости сохраняются во внутреннюю базу данных бота и потом используются для новостных
        рассылок, описанных в разделе \ref{sec:req:fn:newsletter}.

        \todo{Перечень ресурсов, с которых необходимо собирать новости.}

    \subsubsection{Новостная рассылка}
        \label{sec:req:fn:newsletter}
        Один раз в сутки \inlinetodo{возможно, стоит дать пользователю возможность настройки?} бот
        должен рассылать всем пользователям, подписанным на новостную рассылку, новости с внешних
        ресурсов, которые были собраны за этот день. Пользователю
        должна выводиться вся информация, сохранённая в базе данных, перечисленная в разделе
        \ref{sec:req:fn:grabnews}. Если текст новости длиннее 3500~символов или не помещается в
        стандартное сообщение мессенджера, допускается приводить только его начало, указывая,
        что по ссылке на источник откроется полный текст новости.

        Каждая новость присылается одним сообщением, не содержит вложений и
        должна быть маркирована хэштегом \hbox{\texttt{\#новость}}.

        Пользователям, не подписанным на новостную рассылку на момент её проведения, не должны
        рассылаться новостные сообщения.

        Бот не должен присылать сообщения с новостями пользователям, которые в данный момент находятся
        в интерактивном состоянии (см. раздел \ref{seq:req:ui:states}),
        однако он может уведомить их о том, что новостная рассылка для них готова.
        После выхода пользователя из интерактивного состояния бот должен повторить попытку
        рассылки не ранее, чем через 30~секунд, но не позднее, чем через 30~минут.
        Если, руководствуясь данными правилами, бот не может отправить пользователю рассылку
        более 24~часов, то допускается не отправлять данную рассылку этому пользователю в этот день.

    \subsubsection{Сервисные уведомления}
        \label{sec:req:fn:service}
        При возникновении определённых ситуаций, описанных в техническом задании,
        бот рассылает сервисные уведомления тем пользователям, которые на них подписаны и имеют
        право их получать. Сервисное уведомление представляет из себя одно сообщение с текстом
        не длиннее 3500~символов и не более, чем одним вложением.

        Каждое сервисное уведомление должно быть промаркировано хэштегом \hbox{\texttt{\#сервисное}}.
        Также применимы правила отправки рассылок пользователям в интерактивном состоянии,
        описанные в разделе \ref{sec:req:fn:newsletter}.

    \subsubsection{Управление подпиской на рассылки}
        \label{sec:req:fn:subscriptions}
        Любой пользователь имеет возможность подписаться, отписаться и проверить статус подписки на
        любые рассылки, доступные этому пользователю. Перечень рассылок, поддерживаемых ботом:
        \begin{enumerate}
            \item
                Новостная рассылка, описанная в разделе \ref{sec:req:fn:newsletter}.
                Доступна всем пользователям, по умолчанию подписка неактивна.
            \item
                Рассылка сервисных уведомлений, описанная в разделе \ref{sec:req:fn:service}.
                Доступна только пользователям, наделённых правом получения сервисных уведомлений,
                подписка автоматически становится активна, когда пользователя в первый
                раз наделяют необходимыми правами.
        \end{enumerate}

    \subsubsection{Управление правами пользователей}
        \label{seq:req:fn:roles}
        Пользователи, наделённые правами администратора, могут выдавать другим
        пользователям права и отзывать их. Также они могут редактировать свои права.
        Поддерживаемые права пользователей описаны в разделе \ref{sec:req:sec}.
        После любого редактирования прав пользователей должен остаться хотя бы один
        пользователь с правами администратора. Попытка изменения прав, не соблюдающая
        это ограничение, заканчивается ошибкой и отсутствием изменений в правах.

\subsection{Требования к интерфейсу}
    \label{sec:req:ui}
    Интерфейс взаимодействия с ботом (отправка текста, изображений, видео и файлов)
    должен обеспечиваться стандартными средствами мессенджера.

\subsection{Требования к безопасности}
    \label{sec:req:sec}
    \todo{Описание прав пользователей и управления ими.}

\subsection{Требования к надёжности}
    \label{sec:req:reliab}
    \todo{Требования к обеспечению надёжного функционирования,
    контроль входных и выходных данных, время восстановления после отказа, etc.}

\subsection{Условия эксплуатации}
    \label{sec:req:maint}
    \todo{Физические условия, вид обслуживания, количество и квалификация персонала для обслуживания}

\subsection{Требования к составу и параметрам технических средств}
    \label{sec:req:hw}
    \todo{Требования к серверу (аппаратные и программные) и к клиентским приложениям для мессенджера}

\subsection{Требования к информационной и программной совместимости}
    \label{sec:req:compat}
    \todo{Формат данных, методы решения, языки программирования, библиотеки, etc.}

\subsection{Требования к маркировке и упаковке}
    \label{sec:req:ship}
    \todo{Поставка программы заказчику, взаимодействие пользователей с программой}

\subsection{Требования к транспортированию и хранению}
    \label{sec:req:data}
    \todo{Условия хранения и передачи данных}

\subsection{Специальные требования}
    \label{sec:req:etc}
    \todo{Другие требования при наличии}


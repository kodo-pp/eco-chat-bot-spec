\section{Приблизительные макеты экранов интерфейса}

\begin{myfigure}[h!]{fig:sketch:init}
    \messages{0.8\linewidth}{
        \lmessage{
            Вас приветствует HSE EcoBot!

            Для начала работы нажмите кнопку <<Начать>>
        }
        \withbuttons
        \lbuttons{\button{\messagewidth}{Начать}}

        {}
    }
    \caption{Интерфейс в изначальном состоянии.}
\end{myfigure}

\begin{myfigure}[h!]{fig:sketch:mainmenu}
    \messages{0.8\linewidth}{
        \lmessage{
            \textbf{База знаний}

            Коллекция статей и заметок, написанных нами специально для бота%
        }
        \withbuttons\lbuttons{\button{\messagewidth}{\emjfolder{} Сортировка отходов}}
        \withbuttons\lbuttons{\button{\messagewidth}{\emjfolder{} Переработка отходов}}
        \withbuttons\lbuttons{\button{\messagewidth}{\emjfolder{} О боте}}
        \withbuttons\lbuttons{\button{\messagewidth}{Все статьи}}
        \withbuttons\lbuttons{\button{\messagewidth}{Архив рассылок}}

        \lmessage{
            \textbf{Ресурсы}

            Интеграция с важными ресурсами, собранными здесь%
        }
        \withbuttons\lbuttons{\button{\messagewidth}{Зелёная вышка}}
        \withbuttons\lbuttons{\button{\messagewidth}{Индекс качества воздуха}}
        \withbuttons\lbuttons{\button{\messagewidth}{Календарь событий}}

        \lmessage{
            \textbf{Действия с ботом}

            Настройки и служебные действия%
        }
        \withbuttons\lbuttons{\button{\messagewidth}{Управление рассылками}}
        \withbuttons\lbuttons{\button{\messagewidth}{Обратная связь}}
        \withbuttons\lbuttons{\button{\messagewidth}{Предложить экологическую инициативу}}

        {}
    }
    \caption{Интерфейс главного меню для обычных пользователей.}
\end{myfigure}

\begin{myfigure}[h!]{fig:sketch:mainmenu-for-admins}
    \messages{0.8\linewidth}{
        \lmessage{
            \textbf{База знаний}

            Коллекция статей и заметок, написанных нами специально для бота%
        }
        \withbuttons\lbuttons{\button{\messagewidth}{\emjfolder{} Сортировка отходов}}
        \withbuttons\lbuttons{\button{\messagewidth}{\emjfolder{} Переработка отходов}}
        \withbuttons\lbuttons{\button{\messagewidth}{\emjfolder{} О боте}}
        \withbuttons\lbuttons{\button{\messagewidth}{Все статьи}}
        \withbuttons\lbuttons{\button{\messagewidth}{Архив рассылок}}

        \lmessage{
            \textbf{Ресурсы}

            Интеграция с важными ресурсами, собранными здесь%
        }
        \withbuttons\lbuttons{\button{\messagewidth}{Зелёная вышка}}
        \withbuttons\lbuttons{\button{\messagewidth}{Индекс качества воздуха}}
        \withbuttons\lbuttons{\button{\messagewidth}{Календарь событий}}

        \lmessage{
            \textbf{Действия с ботом}

            Настройки и служебные действия%
        }
        \withbuttons\lbuttons{\button{\messagewidth}{Панель администратора}}
        \withbuttons\lbuttons{\button{\messagewidth}{Управление рассылками}}
        \withbuttons\lbuttons{\button{\messagewidth}{Обратная связь}}
        \withbuttons\lbuttons{\button{\messagewidth}{Предложить экологическую инициативу}}

        {}
    }
    \caption{Интерфейс главного меню для пользователей с правами администратора.}
\end{myfigure}

\begin{myfigure}[h!]{fig:sketch:kb-navigation-for-viewers}
    \messages{0.8\linewidth}{
        \lmessage{
            \textbf{База знаний}

            Вы находитесь в разделе <<Сортировка отходов>>

            Подразделы и заметки:%
        }
        \withbuttons\lbuttons{
            \button{0.25\messagewidth}{Вверх}
            \button{0.65\messagewidth}{В главное меню}
        }
        \withbuttons\lbuttons{\button{\messagewidth}{\emjnote{} Карта сбора --- ВШЭ}}
        \withbuttons\lbuttons{\button{\messagewidth}{\emjnote{} Карта сбора --- Москва и МО}}
        \withbuttons\lbuttons{\button{\messagewidth}{\emjnote{} Инструкция по сортировке}}

        {}
    }
    \caption{Интерфейс навигации по разделу базы знаний для обычных пользователей.}
\end{myfigure}

\begin{myfigure}[h!]{fig:sketch:kb-navigation-for-editors}
    \messages{0.8\linewidth}{
        \lmessage{
            \textbf{База знаний}

            Вы находитесь в разделе <<Сортировка отходов>>

            Подразделы и заметки:%
        }
        \withbuttons\lbuttons{
            \button{0.25\messagewidth}{Вверх}
            \button{0.65\messagewidth}{В главное меню}
        }
        \withbuttons\lbuttons{\button{\messagewidth}{Редактировать этот раздел}}
        \withbuttons\lbuttons{\button{\messagewidth}{\emjnote{} Карта сбора --- ВШЭ}}
        \withbuttons\lbuttons{\button{\messagewidth}{\emjnote{} Карта сбора --- Москва и МО}}
        \withbuttons\lbuttons{\button{\messagewidth}{\emjnote{} Инструкция по сортировке}}

        {}
    }
    \caption{Интерфейс навигации по разделу базы знаний для пользователей с правом её редактирования.}
\end{myfigure}

\begin{myfigure}[h!]{fig:sketch:edit-section}
    \messages{0.8\linewidth}{
        \lmessage{
            \textbf{Редактирование раздела <<Сортировка отходов>>}

            Раздел содержит 3 заметки

            Раздел закреплён в главном меню%
        }
        \withbuttons\lbuttons{\button{\messagewidth}{Создать заметку}}
        \withbuttons\lbuttons{\button{\messagewidth}{Создать подраздел}}
        \withbuttons\lbuttons{\button{\messagewidth}{Переименовать}}
        \withbuttons\lbuttons{\button{\messagewidth}{Переместить в другой раздел}}
        \withbuttons\lbuttons{\button{\messagewidth}{Удалить}}
        \withbuttons\lbuttons{\button{\messagewidth}{Открепить из главного меню}}
        \withbuttons\lbuttons{\button{\messagewidth}{Назад}}

        {}
    }
    \caption{Интерфейс редактирования раздела.}
\end{myfigure}

\begin{myfigure}[h!]{fig:sketch:kb-pagination}
    \messages{0.8\linewidth}{
        \lmessage{
            \textbf{База знаний}

            Вы находитесь в разделе <<Тестовый раздел / Вложенный раздел>>

            Подразделы и заметки (показаны 50--59 из 100):%
        }
        \withbuttons\lbuttons{
            \button{0.25\messagewidth}{Вверх}
            \button{0.65\messagewidth}{В главное меню}
        }
        \withbuttons\lbuttons{
            \button{1em}{$\ll$}
            \button{1em}{$<$}
            \button{1em}{$>$}
            \button{1em}{$\gg$}
        }
        \withbuttons\lbuttons{\button{\messagewidth}{\emjnote{} Заметка 50}}
        \withbuttons\lbuttons{\button{\messagewidth}{\emjnote{} Заметка 51}}
        \withbuttons\lbuttons{\button{\messagewidth}{\emjnote{} Заметка 52}}
        \withbuttons\lbuttons{\button{\messagewidth}{\emjnote{} Заметка 53}}
        \withbuttons\lbuttons{\button{\messagewidth}{\emjnote{} Заметка 54}}
        \withbuttons\lbuttons{\button{\messagewidth}{\emjnote{} Заметка 55}}
        \withbuttons\lbuttons{\button{\messagewidth}{\emjnote{} Заметка 56}}
        \withbuttons\lbuttons{\button{\messagewidth}{\emjnote{} Заметка 57}}
        \withbuttons\lbuttons{\button{\messagewidth}{\emjnote{} Заметка 58}}
        \withbuttons\lbuttons{\button{\messagewidth}{\emjnote{} Заметка 59}}

        {}
    }
    \caption{Интерфейс навигации по разделу базы знаний с разбиением выдачи на страницы.}
\end{myfigure}

\begin{myfigure}[h!]{fig:sketch:kb-note-for-viewers}
    \messages{0.8\linewidth}{
        \lmessage{
            \textbf{Тестовая заметка}

            Lorem ipsum dolor sit amet, consectetur adipiscing elit, sed do eiusmod tempor
            incididunt ut labore et dolore magna aliqua. Ut enim ad minim veniam, quis
            nostrud exercitation ullamco laboris nisi ut aliquip ex ea commodo consequat.%
        }
        \withbuttons\lbuttons{
            \button{0.25\messagewidth}{Назад}
            \button{0.65\messagewidth}{В главное меню}
        }

        {}
    }
    \caption{Интерфейс просмотра заметки для пользователей без прав на её редактирование.}
\end{myfigure}

\begin{myfigure}[h!]{fig:sketch:kb-note-for-editors}
    \messages{0.8\linewidth}{
        \lmessage{
            \textbf{Тестовая заметка}

            Lorem ipsum dolor sit amet, consectetur adipiscing elit, sed do eiusmod tempor
            incididunt ut labore et dolore magna aliqua. Ut enim ad minim veniam, quis
            nostrud exercitation ullamco laboris nisi ut aliquip ex ea commodo consequat.%
        }
        \withbuttons\lbuttons{
            \button{\messagewidth}{Редактировать}
        }
        \withbuttons\lbuttons{
            \button{\messagewidth}{Переименовать}
        }
        \withbuttons\lbuttons{
            \button{\messagewidth}{Переместить в другой раздел}
        }
        \withbuttons\lbuttons{
            \button{\messagewidth}{Удалить}
        }
        \withbuttons\lbuttons{
            \button{\messagewidth}{Закрепить в главном меню}
        }
        \withbuttons\lbuttons{
            \button{0.25\messagewidth}{Назад}
            \button{0.65\messagewidth}{В главное меню}
        }

        {}
    }
    \caption{Интерфейс просмотра заметки для пользователей, имеющих право на её редактирование.}
\end{myfigure}

\begin{myfigure}[h!]{fig:sketch:subscriptions}
    \messages{0.8\linewidth}{
        \lmessage{
            \textbf{Подписки на рассылки}

            ~

            \textbf{Новостная рассылка} \\
            Новости об экологии в Вышке и за её пределами \\
            \emjcrossmark{} Вы не подписаны \\
            Подписаться: \fakelink{/sub\_news}

            ~

            \textbf{Обратная связь} \\
            Пожелания и замечания от пользователей бота \\
            \emjcheckmark{} Вы подписаны \\
            Отписаться: \fakelink{/unsub\_feedback}%

            ~

            \textbf{Сервисные уведомления} \\
            Уведомления об ошибках и сбоях в работе бота \\
            \emjcheckmark{} Вы подписаны \\
            Отписаться: \fakelink{/unsub\_service}%
        }
        \withbuttons\lbuttons{\button{\messagewidth}{В главное меню}}

        \rmessage{\fakelink{/sub\_news}}

        \lmessage{Вы подписались на новостную рассылку}

        {}
    }
    \caption{Интерфейс управления подписками на рассылки.}
\end{myfigure}

\begin{myfigure}[h!]{fig:sketch:adminpanel}
    \messages{0.8\linewidth}{
        \lmessage{
            \textbf{Панель администратора}

            Здесь вы можете настроить права пользователей бота%
        }
        \withbuttons\lbuttons{\button{\messagewidth}{Права пользователей}}
        \withbuttons\lbuttons{\button{\messagewidth}{В главное меню}}

        {}
    }
    \caption{Интерфейс панели администратора.}
\end{myfigure}

\begin{myfigure}[h!]{fig:sketch:privs}
    \messages{0.8\linewidth}{
        \lmessage{
            \textbf{Права пользователей}%

            Легенда: \\
            \emjpencil{} Редактирование базы знаний \\
            \emjexclamation{} Получение сервисных уведомлений \\
            \emjspeech{} Получение обратной связи \\
            \emjgear{} Администратор

            ~

            \textbf{\fakelink{@user\_aaa}} (Вы):
                \emjpencil{}\emjexclamation{}\emjspeech{}\emjgear{}
                \fakelink{/u\_17h682f060} \\
            \fakelink{@user\_bbb}:
                \emjpencil{}
                \fakelink{/u\_3cf9g1g86f} \\
            \fakelink{@user\_ccc}:
                \emjexclamation{}\emjspeech{}
                \fakelink{/u\_5695d82869}%
        }
        \withbuttons\lbuttons{\button{\messagewidth}{Выдать права другому пользователю}}
        \withbuttons\lbuttons{\button{\messagewidth}{В панель администратора}}
        \withbuttons\lbuttons{\button{\messagewidth}{В главное меню}}

        \rmessage{\fakelink{/u\_3cf9g1g86f}}

        \lmessage{
            \textbf{Пользователь \fakelink{@user\_bbb}}

            ~

            Права: \\
            \emjpencil{} Редактирование базы знаний%
        }
        \withbuttons\lbuttons{\button{\messagewidth}{
            \emjpencil{} Запретить редактирование базы знаний%
        }}
        \withbuttons\lbuttons{\button{\messagewidth}{
            \emjexclamation{} Разрешить получение сервисных уведомлений%
        }}
        \withbuttons\lbuttons{\button{\messagewidth}{
            \emjspeech{} Разрешить получение обратной связи%
        }}
        \withbuttons\lbuttons{\button{\messagewidth}{
            \emjgear{} Сделать администратором%
        }}
        \withbuttons\lbuttons{\button{\messagewidth}{Назад}}

        {}
    }
    \caption{Интерфейс управления правами пользователей.}
\end{myfigure}

\begin{myfigure}[h!]{fig:sketch:privsnew}
    \messages{0.8\linewidth}{
        \lmessage{
            Введите имя или идентификатор пользователя
            (в формате \texttt{username}, \texttt{@username} или \texttt{id:123456}),
            которому требуется выдать права, либо нажмите <<Назад>> для отмены операции
        }
        \withbuttons\lbuttons{\button{\messagewidth}{Назад}}

        \rmessage{user\_ddd}

        \lmessage{
            \textbf{Пользователь \fakelink{@user\_ddd}}

            ~

            Права: \\
            (пусто)
        }
        \withbuttons\lbuttons{\button{\messagewidth}{
            \emjpencil{} Разрешить редактирование базы знаний%
        }}
        \withbuttons\lbuttons{\button{\messagewidth}{
            \emjexclamation{} Разрешить получение сервисных уведомлений%
        }}
        \withbuttons\lbuttons{\button{\messagewidth}{
            \emjspeech{} Разрешить получение обратной связи%
        }}
        \withbuttons\lbuttons{\button{\messagewidth}{
            \emjgear{} Сделать администратором%
        }}
        \withbuttons\lbuttons{\button{\messagewidth}{Назад}}

        {}
    }
    \caption{Интерфейс выдачи прав пользователям не из списка.}
\end{myfigure}

\begin{myfigure}[h!]{fig:sketch:edit-note}
    \messages{0.8\linewidth}{
        \lmessage{
            \textbf{Редактирование заметки <<Тестовая заметка 2>>}

            ~

            Существующие вложения (1 файл) будут сохранены. Для изменения этого
            поведения нажмите на кнопку <<Не сохранять существующие вложения>>

            ~

            Нажмите <<Назад>> для отмены редактирования%
        }
        \withbuttons\lbuttons{\button{\messagewidth}{Не сохранять существующие вложения}}
        \withbuttons\lbuttons{\button{\messagewidth}{Назад}}

        \rmessage{Новый текст заметки.}

        \lmessage{
            Заметка <<Тестовая заметка 2>> была успешно отредактирована%
        }

        {}
    }
    \caption{Интерфейс редактирования заметки.}
\end{myfigure}

\begin{myfigure}[h!]{fig:sketch:add-note}
    \messages{0.8\linewidth}{
        \lmessage{
            \textbf{Создание новой заметки}

            Введите название новой заметки%
        }
        \withbuttons\lbuttons{\button{\messagewidth}{Назад}}

        \rmessage{Новая заметка}

        \lmessage{
            Введите текст заметки

            Вы также можете приложить к заметке изображения, видео или файлы (не более 10 штук в сумме)%
        }
        \withbuttons\lbuttons{\button{\messagewidth}{Назад}}

        \rmessage{Текст новой заметки.}

        \lmessage{
            Заметка <<Новая заметка>> была создана%
        }

        {}
    }
    \caption{Интерфейс создания новой заметки.}
\end{myfigure}

\begin{myfigure}[h!]{fig:sketch:add-section}
    \messages{0.8\linewidth}{
        \lmessage{
            \textbf{Создание нового раздела}

            Введите название нового раздела%
        }
        \withbuttons\lbuttons{\button{\messagewidth}{Назад}}

        \rmessage{Новый раздел}

        \lmessage{
            Раздел <<Новый раздел>> был успешно создан%
        }

        {}
    }
    \caption{Интерфейс создания новой заметки.}
\end{myfigure}

\begin{myfigure}[h!]{fig:sketch:rename-note}
    \messages{0.8\linewidth}{
        \lmessage{
            \textbf{Переименование заметки <<Тестовая заметка 3>>}

            Введите новое название заметки%
        }
        \withbuttons\lbuttons{\button{\messagewidth}{Назад}}

        \rmessage{Тестовая заметка 33}

        \lmessage{
            Заметка <<Тестовая заметка 3>> была переименована в <<Тестовая заметка 33>>%
        }

        {}
    }
    \caption{Интерфейс переименования заметки.}
\end{myfigure}

\begin{myfigure}[h!]{fig:sketch:move-note}
    \messages{0.8\linewidth}{
        \lmessage{
            \textbf{Перемещение заметки <<Тестовая заметка 4>>}

            Выберите раздел, куда требуется переместить заметку

            ~

            Вы находитесь в корневом разделе

            Подразделы:%
        }
        \withbuttons\lbuttons{
            \button{0.25\messagewidth}{Вверх}
            \button{0.65\messagewidth}{Отменить перемещение}
        }
        \withbuttons\lbuttons{\button{\messagewidth}{Переместить сюда}}
        \withbuttons\lbuttons{\button{\messagewidth}{\emjfolder{} Сбор отходов}}
        \withbuttons\lbuttons{\button{\messagewidth}{\emjfolder{} Переработка отходов}}
        \withbuttons\lbuttons{\button{\messagewidth}{\emjfolder{} О боте}}

        {}
    }
    \caption{Интерфейс перемещения заметки.}
\end{myfigure}

\begin{myfigure}[h!]{fig:sketch:delete-note}
    \messages{0.8\linewidth}{
        \lmessage{
            \textbf{Удаление заметки <<Тестовая заметка 5>>}

            Внимание! Вы действительно хотите удалить заметку <<Тестовая заметка 5>>
            из раздела <<Тестовый раздел 1 / Тестовый раздел 1.3>>?

            Все вложения из этой заметки также будут удалены
        }
        \withbuttons\lbuttons{
            \button{0.45\messagewidth}{Да, удалить}
            \button{0.45\messagewidth}{Нет, не удалять}
        }

        {}
    }
    \caption{Интерфейс подтверждения удаления заметки.}
\end{myfigure}

\begin{myfigure}[h!]{fig:sketch:delete-section}
    \messages{0.8\linewidth}{
        \lmessage{
            \textbf{Удаление раздела <<Тестовый раздел 6>>}

            Внимание! Вы действительно хотите удалить раздел <<Тестовый раздел 6>>
            из корневого раздела?

            Все заметки и разделы в нём будут также удалены. Кроме того, будут удалены все вложения
            из удаляемых заметок
        }
        \withbuttons\lbuttons{
            \button{0.45\messagewidth}{Да, удалить}
            \button{0.45\messagewidth}{Нет, не удалять}
        }

        {}
    }
    \caption{Интерфейс подтверждения удаления раздела.}
\end{myfigure}

\begin{myfigure}[h!]{fig:sketch:feedback}
    \messages{0.8\linewidth}{
        \lmessage{
            \textbf{Обратная связь}

            Здесь вы можете оставить свои пожелания и замечания по нашей работе и по работе бота.
            Также вы можете связаться с нами по email: \fakelink{example@example.com}

            ~

            Выберите тему обратной связи:%
        }
        \withbuttons\lbuttons{\button{\messagewidth}{Зелёная вышка}}
        \withbuttons\lbuttons{\button{\messagewidth}{Чат-бот}}
        \withbuttons\lbuttons{\button{\messagewidth}{Предложить экологическую инициативу}}
        \withbuttons\lbuttons{\button{\messagewidth}{Другое}}
        \withbuttons\lbuttons{\button{\messagewidth}{В главное меню}}

        (выбран вариант <<Другое>>)

        \lmessage{
            \textbf{Обратная связь --- другое}

            Напишите свои пожелания или замечания в сообщении. Вы также можете
            прикрепить изображения, видео или файлы%
        }
        \withbuttons\lbuttons{\button{\messagewidth}{Выбрать другую тему}}
        \withbuttons\lbuttons{\button{\messagewidth}{В главное меню}}

        \rmessage{
            Тестовое сообщение%
        }

        \lmessage{
            Обратная связь отправлена!%
        }

        {}
    }
    \caption{Интерфейс обратной связи.}
\end{myfigure}

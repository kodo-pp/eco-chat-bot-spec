\section{Требования к программной документации}
\label{sec:doc}
Программная документация должна поставляться вместе с исходными текстами бота.
Она должна включать в себя следующие документы:

\begin{enumerate}
    \item
        Руководство пользователя
        
        В руководстве пользователя должны описываться действия, которые пользователь может
        осуществить с ботом, права, которые пользователю для этого необходимы и шаги,
        которые необходимо для этого совершить. Действия, доступные для пользователей
        с разными наборами прав должны быть описаны в одном документе, но могут быть описаны
        в разных его разделах. Данное руководство может предполагать, что пользователь знаком
        с интерфейсом и принципами работы мессенджера и способен самостоятельно использовать
        предоставляемые им функции.

    \item
        Руководство по установке и администрированию

        В данном руководстве должен быть описн процесс установки, обновления и администрирования бота.
        В частности, должны быть указаны:
        \begin{itemize}
            \item
                Список зависимостей и названия пакетов с ними в поддерживаемых дистрибутивах Linux
            \item
                Требования к техническим характеристикам к сервера, не противоречащие требованиям,
                указанным в настоящем техническом задании
            \item
                Инструкция по установке, по обновлению и по удалению в поддерживаемых дистрибуивах Linux
            \item
                Список поставляемых вместе с ботом инструментов администрирования (скриптов, утилит),
                описание действий и инструкция по работе с каждым из них.
        \end{itemize}
\end{enumerate}

Документация должна поставляться в электронном виде в формате PDF и/или в формате HTML.

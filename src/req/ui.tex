\subsection{Требования к интерфейсу}
\label{sec:req:ui}
Интерфейс взаимодействия с ботом (отправка текста, изображений, видео и файлов)
должен обеспечиваться стандартными средствами мессенджера. В любой момент времени работы
бота пользователю доступны:
\begin{enumerate}
    \item
        История взаимодействия с ботом: сообщения, отправленные ботом и пользователем
        друг другу
    \item
        Состояние, в котором на текущий момент находится взаимодействие с ботом, определяемое
        по последнему сообщению, отправленному ботом\footnote{В данном документе встречаются формулировки
        <<состояние пользователя>>, <<состояние бота>> и <<состояние взаимодействия
        (между пользоватем и ботом)>>. Данные формулировки являются эквивалентными и имеют
        один и тот же смысл. Формулировка <<состояние бота>> всегда подразумевает конкретного пользователя,
        взаимодействующего с ботом, поскольку состояние не является глобальным: взаимодействие одного
        пользователя с ботом не влияет на состояние взаимодействия другого пользователя с этим ботом.}
    \item
        Доступные методы дальнейшего взаимодействия с ботом: поле для ввода текста и вложений
        и/или кнопки или ссылки с вариантами взаимодействия, нажатие на которые приведёт к передаче
        команды боту, в зависимости от текущего состояния взаимодействия
\end{enumerate}
В каждом состоянии пользователю доступен определённый набор действий, каждое из которых
приводит к ответу со стороны бота и переходу в другое или то же самое состояние.
Информация, отправляемая и принимаемая ботом зависит от текущего состояния и определяется в разделе
\ref{sec:req:ui:states}.

\subsubsection{Состояния взаимодействия пользователя с ботом}
    \label{sec:req:ui:states}
    \begingroup
    \newcommand{\action}[1]{\textit{#1}}%
    \newcommand{\state}[1]{\uline{#1}}%
    В данном разделе \action{курсивом} указаны действия пользователя, а \state{подчёркнутым текстом}
    --- состояния взаимодействия.

    Пользователь, который до этого не взаимодействовал с ботом, а также пользователь, который
    до этого \action{остановил} бота и ещё не \action{запустил} его вновь, начинают
    своё взаимодействие с ним в \hyperref[itm:req:ui:states:init]{\state{первоначальном}} состоянии.
    Дальнейшее взаимодействие происходит согласно списку состояний, указанному ниже.
    Если для какого-то состояния какой-то вариант ввода со стороны пользователя не указан в данном
    списке, бот должен в ответ на него отправить сообщение, объясняющее, что данный
    ввод некорректен и остаться в том же состоянии.

    Интерактивным называется состояние, в котором пользователь не находится в процессе выполнения
    какого-либо действия с ботом. Для каждого из перечисленных ниже состояний может быть указано,
    считается ли оно интерактивным или нет. Если для какого-то состояния такого указания нет,
    оно считается интерактивным.
    \begin{enumerate}
        \item \label{itm:req:ui:states:init}
            \state{Первоначальное} состояние \\
            Состояние, в котором пользователь ещё не начал общение с ботом.
            Пользователю должна быть доступна кнопка с текстом <<Начать>>\footnote{
            Везде в этом документе, где указан текст элементов интерфейса, допускается
            использование другого текста с таким же смыслом, если это не повредит восприятию
            интерфейса пользователем.}, за исключением случаев, когда это технически невозможно.

            Любой ввод со стороны пользователя кроме отправки сообщений с одним или несколькими
            вложениями должен восприниматься ботом как \action{запуск} бота, при этом бот должен прислать
            сообщение с приветствием и кратким описанием своих функций, и перейти в состояние
            \hyperref[itm:req:ui:states:mainmenu]{\state{главного меню}}.
            Если же пользователь отправляет сообщение с одним или несколькими
            вложениями, бот должен отправить сообщение с объяснением, что он не готов принять вложения
            в данный момент, и что для того, чтобы начать работу с ботом, требуется \action{запустить}
            его. Состояние при этом не должно измениться.

            Данное состояние не считается интерактивным.

        \item \label{itm:req:ui:states:mainmenu}
            \state{Главное меню} \\
            Состояние, в котором пользователь оказывается непосредственно после \action{запуска}
            бота или после завершения какого-либо действия с ботом. Пользователю должны быть доступны
            следующие интерактивные кнопки или ссылки:
            \begin{itemize}
                \item
                    По одной кнопке на каждый закреплённый раздел и каждую закреплённую заметку.
                    При \action{нажатии} на кнопку закреплённого раздела пользователь переходит
                    в состояние
                    \hyperref[itm:req:ui:states:navx]
                    {\state{навигации по базе знаний в \placeholder{выбранном разделе}}}.
                    При \action{нажатии} на кнопку закреплённой заметки пользователь переходит
                    в состояние
                    \hyperref[itm:req:ui:states:view-note]
                    {\state{просмотра \placeholder{выбранной заметки}}}.
                    Кнопки доступны и видимы всем пользователям.
                \item
                    <<Все статьи>>
                    При \action{нажатии} на кнопку пользователь переходит в состояние
                    \hyperref[itm:req:ui:states:navx]
                    {\state{навигации по базе знаний в корневом разделе}}.
                    Кнопка доступна и видима всем пользователям.
                \item
                    <<Архив рассылок>>
                    При \action{нажатии} на кнопку пользователь переходит в состояние
                    \hyperref[itm:req:ui:states:navx]
                    {\state{навигации по базе знаний в разделе <<Архив рассылок>>}}.
                    Кнопка доступна и видима всем пользователям.
                \item
                    По одной кнопке для каждого ресурса, с которыми настроена интеграция.
                    При \action{нажатии} пользователь переходит в состояние
                    \hyperref[itm:req:ui:states:integrationx]
                    {\state{взаимодействия с \placeholder{выбранным ресурсом}}}.
                    Кнопка доступна и видима всем пользователям.
                \item
                    <<Панель администратора>>.
                    При \action{нажатии} пользователь переходит в состояние
                    \hyperref[itm:req:ui:states:adminpanel]
                    {\state{работы с панелью администратора}}.
                    Кнопка доступна и видима пользователям, имеющим права администратора.
                \item
                    <<Подписки на рассылки>>.
                    При \action{нажатии} пользователь переходит в состояние
                    \hyperref[itm:req:ui:states:subscriptions]
                    {\state{управления подписками}}.
                    Кнопка доступна и видима всем пользователям.
                \item
                    <<Обратная связь>>.
                    При \action{нажатии} пользователь переходит в состояние
                    \hyperref[itm:req:ui:states:feedback]
                    {\state{обратной связи}}.
                    Кнопка доступна и видима всем пользователям.
                \item
                    <<Предложить экологическую инициативу>>.
                    При \action{нажатии} пользователь переходит в состояние
                    \hyperref[itm:req:ui:states:feedbackx]
                    {\state{обратной связи с темой <<Предложить экологическую инициативу>>}}.
                    Кнопка доступна и видима всем пользователям.
            \end{itemize}
            Пример возможного расположения элементов пользовательского интерфейса показан на
            рис.~\ref{fig:sketch:mainmenu} и рис.~\ref{fig:sketch:mainmenu-for-admins}.

            Данное состояние не считается интерактивным.

        \item \label{itm:req:ui:states:navx}
            \state{Навигация по базе знаний в разделе \(X\)}

            Состояние, в котором пользователь находится при перемещении по разделам
            базы знаний. Раздел \(X\) --- это любой существующий раздел в базе знаний (включая корневой).
            Пользователю должен выводиться полный путь к разделу (от корневого). Пример
            вывода пути ко вложенному разделу показан на рис.~\ref{fig:sketch:kb-pagination}.

            В интерфейсе должны быть перечислены все разделы и заметки, содержащиеся в разделе
            \(X\). При необходимости следует использовать постраничный вывод этой информации
            (не более 10 материалов на странице), давая пользователю возможность переключаться
            между страницами вывода. В таком случае необходимо указывать, сколько всего материалов
            в данном разделе и какие по номеру материалы отображаются в данный момент.
            Также пользователю должны быть доступны следующие интерактивные кнопки или ссылки:
            \begin{itemize}
                \item
                    <<Вверх>>.
                    При \action{нажатии} пользователь переходит в состояние
                    \hyperref[itm:req:ui:states:navx]
                    {\state{навигации по базе знаний в разделе \(Y\)}}, где \(Y\)
                    --- это родительский раздел \(X\) (т.е. раздел \(X\) принадлежит
                    разделу \(Y\)).
                    Кнопка доступна всем пользователям и видима, если текущий раздел
                    не является корневым.
                \item
                    <<В главное меню>>.
                    При \action{нажатии} пользователь переходит в состояние
                    \hyperref[itm:req:ui:states:mainmenu]
                    {\state{главного меню}}.
                    Кнопка доступна и видима всем пользователям.
                \item
                    <<Редактировать этот раздел>>.
                    При \action{нажатии} пользователь переходит в состояние
                    \hyperref[itm:req:ui:states:edit-section]
                    {\state{редактирования \placeholder{текущего раздела}}}.
                    Кнопка доступна и видима пользователям, имеющим право на редактирования базы знаний,
                    если текущий раздел не является виртуальным и не находится в виртуальном разделе.
                \item
                    По одной кнопке на каждый раздел и каждую заметку, содержащиеся в разделе \(X\).
                    При \action{нажатии} на кнопку раздела пользователь переходит в состояние
                    \hyperref[itm:req:ui:states:navx]
                    {\state{навигации по базе знаний в \placeholder{выбранном разделе}}}.
                    При \action{нажатии} на кнопку заметки пользователь переходит в состояние
                    \hyperref[itm:req:ui:states:view-note]
                    {\state{просмотра \placeholder{выбранной заметки}}}.
                    Кнопки доступны и видимы всем пользователям.
                \item
                    Прочие элементы интерфейса для организации постраничного вывода.
                    При \action{нажатии} пользователь переходит между страницами
                    вывода, не меняя состояния.
                    Кнопка доступна всем пользователям и видима, если полный
                    вывод не умещается на одну страницу.
                    Пример расположения таких элементов интерфейса показан
                    на рис.~\ref{fig:sketch:kb-pagination}.
            \end{itemize}
            Пример возможного расположения элементов пользовательского интерфейса показан на
            рис.~\ref{fig:sketch:kb-navigation-for-viewers} и
            рис.~\ref{fig:sketch:kb-navigation-for-editors}.

        \item \label{itm:req:ui:states:view-note}
            \state{Просмотр заметки \(X\)}

            Состояние, в котором пользователь имеет возможность прочитать текст и/или загрузить
            вложения определённой заметки. \(X\) может быть любой существующей заметкой.

            Бот отправляет пользователю одно или несколько сообщений (согласно правилам, указанным в
            данном техническом задании), в которых содержится следующая информация:
            \begin{itemize}
                \item
                    Название заметки
                \item
                    Текст заметки
                \item
                    Все вложения, принадлежащие данной заметке (при их наличии)
            \end{itemize}

            Пользователю должны быть доступны следующие интерактивные кнопки или ссылки:
            \begin{itemize}
                \item
                    <<Назад>>.
                    При \action{нажатии} пользователь переходит в состояние
                    \hyperref[itm:req:ui:states:navx]
                    {\state{навигации по базе знаний в разделе \(Y\)}}, где \(Y\)
                    --- это родительский раздел по отношению к заметке \(X\)
                    (т.е. заметка \(X\) принадлежит разделу \(Y\)).
                    Кнопка доступна и видима всем пользователям.
                \item
                    <<В главное меню>>.
                    При \action{нажатии} пользователь переходит в состояние
                    \hyperref[itm:req:ui:states:mainmenu]
                    {\state{главного меню}}.
                    Кнопка доступна и видима всем пользователям.
                \item
                    <<Редактировать>>.
                    При \action{нажатии} пользователь переходит в состояние
                    \hyperref[itm:req:ui:states:edit-note]
                    {\state{редактирования заметки}}.
                    Кнопка доступна и видима пользователям, имеющим право на редактирование базы знаний.
                \item
                    <<Переименовать>>.
                    При \action{нажатии} пользователь переходит в состояние
                    \hyperref[itm:req:ui:states:rename-kbo]
                    {\state{переименования материала}}.
                    Кнопка доступна и видима пользователям, имеющим право на редактирование базы знаний.
                \item
                    <<Переместить в другой раздел>>.
                    При \action{нажатии} пользователь переходит в состояние
                    \hyperref[itm:req:ui:states:move-kbo]
                    {\state{перемещения материала}}.
                    Кнопка доступна и видима пользователям, имеющим право на редактирование базы знаний.
                \item
                    <<Удалить>>.
                    При \action{нажатии} пользователь переходит в состояние
                    \hyperref[itm:req:ui:states:delete-kbo]
                    {\state{подтверждения удаления материала}}.
                    Кнопка доступна и видима пользователям, имеющим право на редактирование базы знаний.
                \item
                    <<Закрепить в главном меню>> или <<Открепить из главного меню>>.
                    При \action{нажатии} заметка становится закреплённой в главном меню или теряет этот
                    статус. Если при нажатии кнопки <<Закрепить в главном меню>> лимит на количество
                    закреплённых в главном меню материалов превышается, бот должен сообщить об
                    этом и не изменять статус заметки. В любом случае, состояние пользователя
                    не изменяется.
                    Кнопка доступна и видима пользователям, имеющим право на редактирование базы знаний.
            \end{itemize}

            Пример возможного расположения элементов пользовательского интерфейса показан на
            рис.~\ref{fig:sketch:kb-note-for-viewers} и
            рис.~\ref{fig:sketch:kb-note-for-editors}.

        \item \label{itm:req:ui:states:integrationx}
            \state{Взаимодействие с ресурсом \(X\)}

            Данный класс состояний описан в разделе \ref{par:req:ui:states:integrations}.

        \item \label{itm:req:ui:states:subscriptions}
            \state{Управление подписками}

            В данном состоянии пользователь видит список доступных ему рассылок и подписан ли
            он на каждую из них, а также может подписаться на какие-либо рассылки или отписаться
            от них.

            Для каждой доступной пользователю подписки должны выводиться
            следующие данные:
            \begin{itemize}
                \item
                    Название рассылки
                \item
                    Краткое описание рассылки
                \item
                    Статус подписки пользователя на эту рассылку (подписан или не подписан)
                \item
                    Кнопка или ссылка для подписки или отписки от данной рассылки
            \end{itemize}

            При \action{нажатии} на кнопку или ссылку для подписки на рассылку или отписки от неё
            бот должен прислать пользователю уведомление о соответствующем изменении в статусе
            его подписки.

            Также пользователю должна быть доступна кнопка или ссылка <<В главное меню>>.
            При \action{нажатии} пользователь переходит в состояние
            \hyperref[itm:req:ui:states:mainmenu]
            {\state{главного меню}}.

            Пример возможного расположения элементов пользовательского интерфейса показан на
            рис.~\ref{fig:sketch:subscriptions}.

        \item \label{itm:req:ui:states:adminpanel}
            \state{Работа с панелью администратора}

            В данном состоянии пользователю показывается меню действий, доступных администраторам.
            Пользователю должны быть доступны следующие интерактивные кнопки или ссылки:
            \begin{itemize}
                \item
                    <<В главное меню>>.
                    При \action{нажатии} пользователь переходит в состояние
                    \hyperref[itm:req:ui:states:mainmenu]
                    {\state{главного меню}}.
                \item
                    <<Права пользователей>>.
                    При \action{нажатии} пользователь переходит в состояние
                    \hyperref[itm:req:ui:states:user-privs]
                    {\state{управления правами пользователей}}.
            \end{itemize}

            Все кнопки доступны и видимы только администраторам.

            Пример возможного расположения элементов пользовательского интерфейса показан на
            рис.~\ref{fig:sketch:adminpanel}.

        \item \label{itm:req:ui:states:user-privs}
            \state{Управление правами пользователей}

            В данном состоянии пользователю показывается список всех пользователей
            имеющих специальные права (см. раздел~\ref{sec:req:sec:privs}), список прав каждого
            упомянутого пользователя, а также предоставляется
            возможность выдать пользователям права или отозвать их.
            Пользователю должны быть доступны следующие интерактивные кнопки или ссылки:
            \begin{itemize}
                \item
                    По одной кнопке или ссылке для управления правами каждого пользователя.
                    При \action{нажатии} пользователь переходит в состояние
                    \hyperref[itm:req:ui:states:user-privsx]
                    {\state{управлению правами \placeholder{выбранного пользователя}}}.
                \item
                    <<Выдать права другому пользователю>>.
                    При \action{нажатии} пользователь переходит в состояние
                    \hyperref[itm:req:ui:states:user-privs-unlisted]
                    {\state{управлению правами пользователя не из списка}}.
                \item
                    <<В главное меню>>.
                    При \action{нажатии} пользователь переходит в состояние
                    \hyperref[itm:req:ui:states:mainmenu]
                    {\state{главного меню}}.
                \item
                    <<В панель администратора>>.
                    При \action{нажатии} пользователь переходит в состояние
                    \hyperref[itm:req:ui:states:adminpanel]
                    {\state{панели администратора}}.
            \end{itemize}

            Все кнопки доступны и видимы только администраторам.

            Пример возможного расположения элементов пользовательского интерфейса показан на
            рис.~\ref{fig:sketch:privs}.

        \item \label{itm:req:ui:states:user-privsx}
            \state{Управление правами пользователя \(X\)}

            В данном состоянии пользователю показывается список прав, которые имеет пользователь \(X\)
            и предоставляется возможность выдать ему права или отозвать их.
            Пользователю должны быть доступны следующие интерактивные кнопки или ссылки:
            \begin{itemize}
                \item
                    <<Запретить/Разрешить редактирование базы знаний>>
                    (в зависимости от текущего набора прав).
                    При \action{нажатии} отзывает или выдаёт пользователю \(X\) право на редактирование
                    базы знаний. Состояние не меняется.
                \item
                    <<Запретить/Разрешить получение сервисных уведомлений>>
                    (в зависимости от текущего набора прав).
                    При \action{нажатии} отзывает или выдаёт пользователю \(X\) право на получение
                    сервисных уведомлений. Состояние не меняется.
                \item
                    <<Запретить/Разрешить получение обратной связи>>
                    (в зависимости от текущего набора прав).
                    При \action{нажатии} отзывает или выдаёт пользователю \(X\) право на получение
                    обратной связи. Состояние не меняется.
                \item
                    <<Сделать администратором / Отозвать статус администратора>>
                    (в зависимости от текущего набора прав).
                    При \action{нажатии} отзывает или выдаёт пользователю \(X\) статус администратора.
                    Состояние не меняется. Если статус отозвать невозможно из-за нарушения ограничений,
                    указанных в этом документе, бот сообщает пользователю об ошибке и не изменяет
                    права пользователя \(X\).
                \item
                    <<Назад>>.
                    При \action{нажатии} пользователь переходит в состояние
                    \hyperref[itm:req:ui:states:user-privs]
                    {\state{управления правами пользователей}}.
            \end{itemize}

            Все кнопки доступны и видимы только администраторам.

            Пример возможного расположения элементов пользовательского интерфейса показан на
            рис.~\ref{fig:sketch:privs}.

        \item \label{itm:req:ui:states:user-privs-unlisted}
            \state{Управление правами пользователя не из списка}

            В данном состоянии можно управлять правами пользователя, не указанного в списке,
            описанном в состоянии
            \hyperref[itm:req:ui:states:user-privs]
            {\state{управления правами пользователей}}.
            Для этого пользователь должен отправить боту сообщение с указанием пользователя,
            к которому нужно применить изменения.
            Пользователю должны быть доступны следующие интерактивные кнопки или ссылки:
            \begin{itemize}
                \item
                    <<Назад>>.
                    При \action{нажатии} пользователь переходит в состояние
                    \hyperref[itm:req:ui:states:user-privs]
                    {\state{управления правами пользователей}}.
            \end{itemize}

            Все кнопки доступны и видимы только администраторам.

            Если перед этим не были совершены другие действия, то сообщение от пользователя
            в одном из форматов, указанных ниже, должно быть принято в качестве идентификационной
            информации другого пользователя. Если пользователь не может быть найден по такой
            идентификационной информации, бот выдаёт сообщение об ошибке и не меняет состояние.
            В случае успеха пользователь переходит в состояние
            \hyperref[itm:req:ui:states:user-privsx]
            {\state{управления правами \placeholder{указанного пользователя}}}.

            Возможные форматы сообщения с идентификационной информацией:
            \begin{itemize}
                \item
                    \texttt{\emph{username}}
                \item
                    \texttt{@\emph{username}}
                \item
                    \texttt{id:\emph{code}}
            \end{itemize}
            Здесь \texttt{\emph{username}} --- это строка с именем пользователя, а
            \texttt{\emph{code}} --- это числовой идентификатор пользователя. Подробнее
            об идентификации пользователей написано в разделе~\ref{sec:req:sec:id}.

            Пример возможного расположения элементов пользовательского интерфейса показан на
            рис.~\ref{fig:sketch:privs}.

        \item \label{itm:req:ui:states:feedback}
            \state{Обратная связь}

            В данном состоянии пользователь может выбрать тему обратной связи.
            Пользователю должны быть доступны следующие интерактивные кнопки или ссылки:
            \begin{itemize}
                \item
                    <<В главное меню>>.
                    При \action{нажатии} пользователь переходит в состояние
                    \hyperref[itm:req:ui:states:mainmenu]
                    {\state{главного меню}}.
                    Кнопка доступна и видима всем пользователям.
                \item
                    По одной кнопке для каждой доступной темы обратной связи
                    (см. раздел~\ref{sec:req:fn:feedback}).
                    При \action{нажатии} пользователь переходит в состояние
                    \hyperref[itm:req:ui:states:feedbackx]
                    {\state{обратной связи с \placeholder{выбранной темой}}}.
                    Кнопка доступна и видима всем пользователям.
            \end{itemize}

            Пример возможного расположения элементов пользовательского интерфейса показан на
            рис.~\ref{fig:sketch:feedback}.

        \item \label{itm:req:ui:states:feedbackx}
            \state{Обратная связь с темой \(X\)}

            В данном состоянии пользователь может написать сообщение с обратной связью
            с выбранной темой. \(X\) может быть любой доступной темой обратной связи
            (см. раздел~\ref{sec:req:fn:feedback}).
            Пользователю должны быть доступны следующие интерактивные кнопки или ссылки:
            \begin{itemize}
                \item
                    <<В главное меню>>.
                    При \action{нажатии} пользователь переходит в состояние
                    \hyperref[itm:req:ui:states:mainmenu]
                    {\state{главного меню}}.
                    Кнопка доступна и видима всем пользователям.
            \end{itemize}

            Если перед этим не были совершены другие действия, то любое сообщение от пользователя,
            возможно, включающее вложения, должно быть принято в качестве содержимого обратной связи,
            если оно не нарушает установленные в этом документе лимиты. В случае успеха
            пользователь переходит в состояние
            \hyperref[itm:req:ui:states:mainmenu]
            {\state{главного меню}}.

            Пример возможного расположения элементов пользовательского интерфейса показан на
            рис.~\ref{fig:sketch:feedback}.

        \item \label{itm:req:ui:states:edit-note}
            \state{Редактирование заметки}

            В данном состоянии пользователь может отправить сообщение с новым содержимым
            заметки, а также указать, что делать с существующими вложениями.
            Пользователю должны быть доступны следующие интерактивные кнопки или ссылки:
            \begin{itemize}
                \item
                    <<Сохранить существующие вложения>> или <<Не сохранять существующие вложения>>.
                    При \action{нажатии} переключается поведение существующих вложений: при настройке
                    <<сохранить>> они будут сохранены, а новые вложения (при наличии)
                    --- добавлены к ним (при условии соблюдения лимита на количество вложений);
                    при настройке <<не сохранять>> существующие вложения будут удалены, а новые
                    вложения (при наличии) их заменят.
                    В любом случае, состояние пользователя не изменяется.
                    Кнопка доступна и видима всем пользователям. Текст варьируется в зависимости от
                    текущей настройки: <<Сохранить существующие вложения>> показывается, когда настройка
                    выставлена в положение <<не сохранять>>, и наоборот.
                \item
                    <<Назад>>.
                    При \action{нажатии} пользователь переходит в состояние
                    \hyperref[itm:req:ui:states:view-note]
                    {\state{просмотра \placeholder{текущей заметки}}}.
                    Кнопка доступна и видима всем пользователям.
            \end{itemize}

            Если перед этим не были совершены другие действия, то любое сообщение от пользователя,
            возможно, включающее вложения, должно быть принято в качестве нового содержимого заметки,
            если оно не нарушает установленные в этом документе лимиты. В случае успеха
            пользователь переходит в состояние
            \hyperref[itm:req:ui:states:view-note]
            {\state{просмотра \placeholder{текущей заметки}}}.

            Пример возможного расположения элементов пользовательского интерфейса показан на
            рис.~\ref{fig:sketch:edit-note}.

        \item \label{itm:req:ui:states:edit-section}
            \state{Редактирование раздела}

            В данном состоянии пользователь может изменить имя, расположение и содержимое не-виртуального
            раздела.
            Пользователю должны быть доступны следующие интерактивные кнопки или ссылки:
            \begin{itemize}
                \item
                    <<Создать заметку>>.
                    При \action{нажатии} пользователь переходит в состояние
                    \hyperref[itm:req:ui:states:create-note]
                    {\state{создания заметки}}.
                \item
                    <<Создать подраздел>>.
                    При \action{нажатии} пользователь переходит в состояние
                    \hyperref[itm:req:ui:states:create-section]
                    {\state{создания раздела}}.
                \item
                    <<Переименовать>>.
                    При \action{нажатии} пользователь переходит в состояние
                    \hyperref[itm:req:ui:states:rename-kbo]
                    {\state{переименования материала}}.
                \item
                    <<Переместить в другой раздел>>.
                    При \action{нажатии} пользователь переходит в состояние
                    \hyperref[itm:req:ui:states:move-kbo]
                    {\state{перемещения материала}}.
                \item
                    <<Удалить>>.
                    При \action{нажатии} пользователь переходит в состояние
                    \hyperref[itm:req:ui:states:delete-kbo]
                    {\state{удаления материала}}.
                \item
                    <<Закрепить в главном меню>> или <<Открепить из главного меню>>.
                    При \action{нажатии} раздел становится закреплённым в главном меню или теряет этот
                    статус. Если при нажатии кнопки <<Закрепить в главном меню>> лимит на количество
                    закреплённых в главном меню материалов превышается, бот должен сообщить об
                    этом и не изменять статус раздела. В любом случае, состояние пользователя
                    не изменяется.
                \item
                    <<Назад>>.
                    При \action{нажатии} пользователь переходит в состояние
                    \hyperref[itm:req:ui:states:navx]
                    {\state{навигации по базе знаний в \placeholder{текущем разделе}}}.
                    Кнопка доступна и видима всем пользователям.
            \end{itemize}

            Все кнопки доступны и видимы только пользователям, имеющим право на редактирование
            базы знаний.

            Пример возможного расположения элементов пользовательского интерфейса показан на
            рис.~\ref{fig:sketch:edit-section}.

        \item \label{itm:req:ui:states:rename-kbo}
            \state{Переименование материала}

            В данном состоянии пользователь может отправить сообщение с новым именем для
            текущей заметки или текущего раздела, чтобы переименовать эту заметку или этот раздел.
            Пользователю должны быть доступны следующие интерактивные кнопки или ссылки:
            \begin{itemize}
                \item
                    <<Назад>>.
                    При \action{нажатии} пользователь переходит в состояние
                    \hyperref[itm:req:ui:states:navx]
                    {\state{навигации по базе знаний в \placeholder{текущем разделе}}}.
                    Кнопка доступна и видима всем пользователям.
            \end{itemize}
            Если перед этим не были совершены другие действия, то сообщение от пользователя
            с новым названием раздела или заметки без вложений должно быть принято,
            и раздел или заметка должны быть переименованы соответствующим образом,
            если данное имя допустимо.
            В случае успеха пользователь переходит в состояние
            \hyperref[itm:req:ui:states:navx]
            {\state{навигации по базе знаний в \placeholder{текущем разделе}}}.

            Все кнопки и действия доступны и видимы только пользователям, имеющим право на редактирование
            базы знаний.

        \item \label{itm:req:ui:states:move-kbo}
            \state{Перемещение материала (текущий раздел \(X\))}

            Состояние, в котором пользователь выбирает раздел, в который следует переместить
            выбранную заметку или раздел, и сейчас находится в разделе \(X\).
            Раздел \(X\) --- это любой существующий раздел в базе знаний (включая корневой).
            Пользователю должен выводиться полный путь к разделу (от корневого). Пример
            вывода пути ко вложенному разделу показан на рис.~\ref{fig:sketch:kb-pagination}.
            При соблюении остальных требований данного пункта, интерфейс в данном состоянии
            должен быть как можно более похожим стилистически и визуально на интерфейс
            в состоянии
            \hyperref[itm:req:ui:states:navx]
            {\state{навигации по базе знаний в разделе \(X\)}}.

            В интерфейсе должны быть перечислены все разделы, содержащиеся в разделе
            \(X\). При необходимости следует использовать постраничный вывод этой информации
            (не более 10 элементов на странице), давая пользователю возможность переключаться
            между страницами вывода. В таком случае необходимо указывать, сколько всего подразделов
            в данном разделе и какие по номеру подраздел отображаются в данный момент.
            Также пользователю должны быть доступны следующие интерактивные кнопки или ссылки:
            \begin{itemize}
                \item
                    <<Вверх>>, по одной кнопке на каждый подраздел, а также
                    элементы интерфейса для организации потраничного вывода при необходимости.
                    Кнопки работают аналогично соответствующим кнопкам из состояния 
                    \hyperref[itm:req:ui:states:navx]
                    {\state{навигации по базе знаний в разделе \(X\)}}, за исключением того,
                    что вместо состояния
                    \hyperref[itm:req:ui:states:navx]
                    {\state{навигации по базе знаний в разделе \(Y\)}}.
                    текущее состояние меняется на состояние
                    \hyperref[itm:req:ui:states:move-kbo]
                    {\state{перемещения материала (текущий раздел \(Y\))}}.
                \item
                    <<Отменить перемещение>>.
                    При \action{нажатии} процесс перемещения материала отменяется.
                    Пользователь возвращается в состояние
                    \hyperref[itm:req:ui:states:navx]
                    {\state{навигации по базе знаний в разделе \(X\)}}.
                \item
                    <<Переместить сюда>>.
                    При \action{нажатии} бот перемещает выбранный материал в выбранный раздел,
                    и сообщает пользователю о результате операции. В любом случае, пользователь
                    переходит в состояние
                    \hyperref[itm:req:ui:states:navx]
                    {\state{навигации по базе знаний в разделе \(N\)}}.
                    В случае успешного перемещения раздел \(N\) --- это перемещённый раздел
                    в новом месте. В случае неуспешного перемещения раздел \(N\) --- это раздел \(X\)
                    без каких-либо изменений.
            \end{itemize}

            Все кнопки доступны и видимы только пользователям, имеющим право на редактирование базы
            знаний.

            Пример возможного расположения элементов пользовательского интерфейса показан на
            рис.~\ref{fig:sketch:move-note}.


        \item \label{itm:req:ui:states:delete-kbo}
            \state{Подтверждение удаления материала}

            В данном состоянии пользователь подтверждает или отменяет удаление выбранного
            материала. Пользователю обязательно должен отображаться полный путь к материалу
            на удаление.
            \begin{itemize}
                \item
                    <<Удалить>>.
                    При \action{нажатии} выбранный материал (и все вложенные в него материалы,
                    если это раздел) удаляются из базы знаний без штатной возможности восстановления.
                    При возникновении ошибки бот должен вывести её пользователю и немеденно прервать
                    операцию удаления.
                    При успешном удалении пользователь переходит в состояние
                    \hyperref[itm:req:ui:states:navx]
                    {\state{навигации по базе знаний в разделе \(Y\)}},
                    где \(Y\) --- это раздел, в котором непосредственно содержался удалённый раздел.
                    При ошибке удаления пользователь переходит в состояние
                    \hyperref[itm:req:ui:states:navx]
                    {\state{навигации по базе знаний в \placeholder{текущем разделе}}}.
                \item
                    <<Не удалять>>.
                    При \action{нажатии} операция удалений материала отменяется.
                    Пользователь переходит в состояние
                    \hyperref[itm:req:ui:states:navx]
                    {\state{навигации по базе знаний в \placeholder{текущем разделе}}}.
            \end{itemize}

            Пример возможного расположения элементов пользовательского интерфейса показан на
            рис.~\ref{fig:sketch:delete-note} и рис.~\ref{fig:sketch:delete-section}.

        \item \label{itm:req:ui:states:create-note}
            \state{Создание заметки}

            В данном состоянии пользователь начинает процесс создания новой заметки и может
            отправить сообщение с именем создаваемой заметки.

            Также пользователю должны быть доступны следующие интерактивные кнопки или ссылки:
            \begin{itemize}
                \item
                    <<Назад>>.
                    При \action{нажатии} пользователь отменяет создание заметки и
                    переходит в состояние
                    \hyperref[itm:req:ui:states:navx]
                    {\state{навигации по базе знаний в \placeholder{текущем разделе}}}.
            \end{itemize}
            
            Если перед этим не были совершены другие действия, то сообщение от пользователя
            с названием новой заметки без вложений должно быть принято ботом, и
            состояние взаимодействия должно измениться на
            \hyperref[itm:req:ui:states:create-notex]
            {\state{создание заметки с названием \placeholder{из этого сообщения}}}.

            Все кнопки и действия доступны и видимы только пользователям, имеющим право на редактирование
            базы знаний.

            Пример возможного расположения элементов пользовательского интерфейса показан на
            рис.~\ref{fig:sketch:add-note}.

        \item \label{itm:req:ui:states:create-notex}
            \state{Создание заметки с названием \(X\)}

            В данном состоянии пользователь продолжает процесс создания новой заметки и может
            отправить сообщение с содержанием заметки.

            Также пользователю должны быть доступны следующие интерактивные кнопки или ссылки:
            \begin{itemize}
                \item
                    <<Назад>>.
                    При \action{нажатии} пользователь возвращается к шагу выбора имени для заметки
                    в состояние
                    \hyperref[itm:req:ui:states:create-note]
                    {\state{создания заметки}}.
            \end{itemize}
            
            Если перед этим не были совершены другие действия, то сообщение от пользователя
            с должно быть принято ботом, и новая заметка должна сохраниться.
            Бот должен сообщить пользователю об успехе или ошибке операции, и
            состояние взаимодействия должно измениться на состояние
            \hyperref[itm:req:ui:states:navx]
            {\state{навигации по базе знаний в \placeholder{текущем разделе}}}.

            Все кнопки и действия доступны и видимы только пользователям, имеющим право на редактирование
            базы знаний.

            Пример возможного расположения элементов пользовательского интерфейса показан на
            рис.~\ref{fig:sketch:add-note}.

        \item \label{itm:req:ui:states:create-section}
            \state{Создание раздела}

            В данном состоянии пользователь может отправить сообщение с именем раздела и таким образом
            создать пустой раздел с выбранным именем.

            Также пользователю должны быть доступны следующие интерактивные кнопки или ссылки:
            \begin{itemize}
                \item
                    <<Назад>>.
                    При \action{нажатии} пользователь отменяет создание раздела и
                    переходит в состояние
                    \hyperref[itm:req:ui:states:navx]
                    {\state{навигации по базе знаний в \placeholder{текущем разделе}}}.
            \end{itemize}
            
            Если перед этим не были совершены другие действия, то сообщение от пользователя
            с названием новой заметки без вложений должно быть принято ботом, и
            бот должен создать в текущем разделе пустой подраздел с таким именем.
            Об успехе или ошибке операции необходимо сообщить пользователю.
            После выполнения операции состояние должно измениться на
            состояние \hyperref[itm:req:ui:states:navx]
            {\state{навигации по базе знаний в \placeholder{текущем разделе}}}.

            Все кнопки и действия доступны и видимы только пользователям, имеющим право на редактирование
            базы знаний.

            Пример возможного расположения элементов пользовательского интерфейса показан на
            рис.~\ref{fig:sketch:add-section}.
    \end{enumerate}

    При слишком долгом нахождении в каком-либо интерактивном состоянии без какой-либо активности
    со стороны пользователя, взаимодействие с ним автоматически переходит в состояние
    \hyperref[itm:req:ui:states:mainmenu]
    {\state{главного меню}}, при этом отменяются все начатые, но не завершённые операции.
    Время, которое должно пройти с момента последней активности пользователя до
    автоматического перехода в главное меню называется временем ожидания. Во всех интерактивных
    состояниях время ожидания составляет один час (3600~секунд), со следующими исключениями:
    \begin{itemize}
        \item
            В состоянии
            \hyperref[itm:req:ui:states:edit-note]
            {\state{редактирования заметки}}
            время ожидания состовляет 6~часов
        \item
            В состоянии
            \hyperref[itm:req:ui:states:create-note]
            {\state{создания заметки с именем \(X\)}}
            время ожидания состовляет 6~часов
    \end{itemize}

    \paragraph{Интеграция с внешними ресурсами}
        \label{par:req:ui:states:integrations}
        ~\par
        В данном разделе описывается пользовательский интерфейс для интеграции с внешними
        ресурсами и сопутствующие состояния взаимодействия с ботом.
        Интерфейс для интеграции с каждым ресурсом описан в своём подразделе.
        Каждый подраздел содержит описание состояний взаимодействия пользователя с ботом,
        связанных с одним ресурсом.

        \subparagraph{Зелёная вышка}
            \begin{itemize}
                \item
                    \todo{\state{Взаимодействие с ресурсом <<Зелёная вышка>>}}
            \end{itemize}

        \subparagraph{Качество воздуха в Москве}
            \begin{itemize}
                \item
                    \todo{\state{Взаимодействие с ресурсом <<Индекс качества воздуха в Москве>>}}
            \end{itemize}

        \subparagraph{Календарь событий}
            \begin{itemize}
                \item
                    \todo{\state{Взаимодействие с ресурсом <<Календарь событий>>}}
            \end{itemize}
    \endgroup

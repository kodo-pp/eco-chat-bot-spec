\subsection{Требования к безопасности}
\label{sec:req:sec}
\subsubsection{Основные положения}
    \label{sec:req:sec:common}
    Бот не должен отправлять или предоставлять пользователям информацию,
    к которой они не имеют доступа, за исключением случаев,
    когда пользователь, имеющий доступ к этой информации, целенаправленно
    отправляет или предоставляет эту информацию другим пользователям.
    Также бот не должен принимать и выполнять или пытаться выполнить какие-либо команды от
    пользователей, которые не имеют права отдавать эти команды.

\subsubsection{Специальные права пользователей}
    \label{sec:req:sec:privs}
    В чат-боте у каждого пользователя есть набор прав. Каждая из приведённых ниже привелегий
    контролирует часть информации и действий, доступных пользователю,
    может быть выдана пользователю или отозвана у него независимо от остальных привелегий
    и других пользователей.

    По умолчанию все пользователи не имеют никаких специальных привелегий. Специальные привелегии
    должно быть возможно выдать пользователю или определённому кругу пользователей при установке.

    Определены следующие привелегии:
    \begin{enumerate}
        \item \label{itm:req:sec:privs:kbedit}
            Право на редактирование базы знаний

            Позволяет пользователям вносить любые возможные изменения в базу знаний,
            в частности создавать, редактировать, перемещать, переименовывать, удалять
            и закреплять/откреплять заметки и не-виртуальные разделы.

        \item \label{itm:req:sec:privs:service}
            Право на получение сервисных уведомлений

            Даёт пользователю возможность получать рассылку сервисных уведомлений и открывает ему доступ
            в раздел <<Сервисные уведомления>> архива рассылок.

        \item \label{itm:req:sec:privs:feedback}
            Право на получение обратной связи

            Даёт пользователю возможность получать рассылку сообщений с обратной связью и
            открывает ему доступ в раздел <<Обратная связь>> архива рассылок.

        \item \label{itm:req:sec:privs:admin}
            Права администратора

            Даёт пользователю возможность выдавать себе и другим пользователям и
            отзывать у них права.

        \item \label{itm:req:sec:privs:calendar}
            Право на управление событиями

            Даёт пользователю возможность создавать, редактировать и удалять события в календаре событий.

        \item \label{itm:req:sec:privs:announce}
            Право на рассылку объявлений

            Даёт пользователю право рассылать объявления всем подписанным на них пользователям.
    \end{enumerate}

\subsubsection{Идентификация пользователей}
    \label{sec:req:sec:id}
    Идентификация пользователей может проходить двумя взаимозаменяемыми способами:
    по имени пользователя в мессенджере, при его наличии, и по внутреннему идентификатору,
    во всех случаях. Аутентификация пользователей, то есть подтверждение корректности и
    принадлежности им идентификационной информации осуществляется мессенджером.

    Имя пользователя --- это строка с глобальным и общедоступным именем пользователя в месенджере.
    Бот должен принимать слеующие форматы этой строки:
    \begin{itemize}
        \item
            \texttt{@\textit{username}}
        \item
            \texttt{\textit{username}}
    \end{itemize}
    
    Внутренний идентификатор --- это числовой идентификатор пользователя, предоставляемый
    програмным интерфейсом мессенджера. Число, иденцифицирующее пользователя, является целым,
    неотриацтельным и меньшим, чем \(2^{64}\). Данный идентификатор записывается в формате
    \texttt{id:\textit{123456}}, где вместо \texttt{\textit{123456}} подставляется соотвтествующий
    пользователю числовой идентификатор.

    Допускаются все перечисленные варианты идентификации пользователей, однако со своей стороны боту
    следует использовать имя пользователя в формате \texttt{@\textit{username}} в приоритетном порядке,
    а если имя пользователя отсутствует --- то внутренний идентификатор.

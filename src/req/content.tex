\subsection{Требования к содержимому базы знаний}
При развёртывании бота на предоставленном Заказчиком сервере
необходимо обеспечить наполнение его базы знаний согласно схеме, приведённой на рис.~\ref{fig:req:content:tree}.
Допускается также создавать дополнительные разделы и заметки, если они будут релевантны.

\begin{figure}[h!]
    \centering
    \setcounter{treeline}0
    \begin{tikzpicture}
        \mynode 0 {\emjfolder{} Корневой раздел}
            \mynode 1 {\emjfolder{} Сортировка отходов}
                \mynode 2 {\emjfolder{} Карта сбора отходов ВШЭ}
                    \mynode 3 {\emjnote{} Карта сбора отходов в НИУ ВШЭ}
                \mynode 2 {\emjfolder{} Карта сбора отходов в Москве и МО}
                    \mynode 3 {\emjnote{} Карта сбора отходов в Москве и МО}
                \mynode 2 {\emjfolder{} Инструкции по сортировке}
                    \mynode 3 {\emjnote{} Пластик}
                    \mynode 3 {\emjnote{} Стекло}
                    \mynode 3 {\emjnote{} Бумага}
                    \mynode 3 {\emjnote{} Металл}
                    \mynode 3 {\emjnote{} Батарейки}
                \mynode 2 {\emjfolder{} Чек-листы}
                    \mynode 3 {\emjnote{}}
                    \mynode 3 {\emjnote{}}
                    \mynode 3 {\emjnote{}}
            \mynode 1 {\emjfolder{} Переработка отходов}
                \mynode 2 {\emjfolder{} Технологии и методы переработки}
                    \mynode 3 {\emjnote{} Технологии и методы переработки}
                \mynode 2 {\emjfolder{} Переработка отходов в Москве}
                    \mynode 3 {\emjnote{} Переработка отходов в Москве}
                \mynode 2 {\emjfolder{} Виды переработки}
                    \mynode 3 {\emjnote{}}
                    \mynode 3 {\emjnote{}}
                    \mynode 3 {\emjnote{}}
                \mynode 2 {\emjfolder{} Заводы}
                    \mynode 3 {\emjnote{} Заводы}
            \mynode 1 {\emjfolder{} Углеродный след}
                \mynode 2 {\emjnote{} Об углеродном следе}
                \mynode 2 {\emjnote{} Как уменьшить свой углеродный след}
                \mynode 2 {\emjnote{} Калькулятор углеродного следа}
            \mynode 1 {\emjfolder{} Зелёная вышка}
                \mynode 2 {\emjnote{} О Зелёной вышке}
                \mynode 2 {\emjnote{} Страница Зелёной вышки}
                \mynode 2 {\emjnote{} Хочу стать волонтёром}
            \mynode 1 {\emjfolder{} Индекс качества воздуха}
                \mynode 2 {\emjnote{} Индекс качества воздуха в Москве}
                \mynode 2 {\emjnote{} Страница AccuWeather}
        %\draw \outpoint00 -- \midpoint80;
        %    \draw \midpoint10 -- \inpoint11;
        %    \draw \midpoint60 -- \inpoint61;
        %    \draw \midpoint80 -- \inpoint81;
        %\draw \outpoint11 -- \midpoint51;
        %    \draw \midpoint21 -- \inpoint22;
        %    \draw \midpoint51 -- \inpoint52;
        %\draw \outpoint22 -- \midpoint42;
        %    \draw \midpoint32 -- \inpoint33;
        %    \draw \midpoint42 -- \inpoint43;
        %\draw \outpoint61 -- \midpoint71;
        %    \draw \midpoint71 -- \inpoint72;
    \end{tikzpicture}
    \caption{Схема требуемого дерева разделов.}
    \label{fig:req:content:tree}
\end{figure}

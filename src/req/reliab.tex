\subsection{Требования к надёжности}
\label{sec:req:reliab}
Бот должен выполнять следующие требования к надёжности, за исключением случаев,
когда выполнить их невозможно технически или юридически:
\begin{enumerate}
    \item
        На любое сообщение или нажатие на кнопку со стороны пользователя бот должен ответить
        своим сообщением. Допускаются также иные способы ответного взаимодействия, предусмотренные
        мессенджером, если факт ответного взаимодействия будет понятен пользователю.
    \item
        При перезапуске бота по любой причине бот должен сохранять неизменной относительно
        состояния непосредственно до перезапуска:
        \begin{itemize}
            \item
                Список пользователей, запустивших бота
            \item
                Права, выданные пользователями
            \item
                Статус подписок пользователей на рассылки
            \item
                Структура и материалы базы знаний
            \item
                Сохранённые новости, которые ещё не были отправлены в рассылку
        \end{itemize}

        Состояния взаимодействия с пользователями должны либо остаться без изменений, либо перейти
        в состояние главного меню для всех пользователей, запустивших бота, если
        пользователи будут уведомлены об изменении состояния.

        Остальная информация может быть потеряна, но не может оказаться в некорректном состоянии.

        Данное требование не может быть выполнено технически при осуществлении внешних модификаций базы
        данных и программного кода бота в процессе его работы или до его повторного запуска.
        Однако если база данных бота изменена поставляющимися с ним инструментами (скриптами, утилитами),
        то данное требование продолжает применяться, однако внесённые инструментами изменения должны
        быть приняты ботом.

        Часть информации (например, загруженные пользователями вложения) допустимо хранить в
        инфраструктуре мессенджера. В таком случае не даётся никаких гарантий по их сохранности,
        за исключением гарантий, предоставляемых мессенджером. Отсутствие доступа
        к данной информации со стороны бота не должно приводить к невозможности использования
        других его функций, не связанных с этой информацией.
    \item
        Функционирование бота может зависеть от корректной работы операционной системы сервера,
        его аппаратного обеспечения, сетевой инфраструктуры сервера и сети Интернет и систем мессенджера.
        При некорректной работе этих систем бот не обязан корректно выполнять функции и требования,
        исполнение которых зависит от них.
    \item
        При возникновении по любой причине некорректных данных в боте, бот должен предпринять
        разумные меры по обходу этой ситуации с целью продолжения предоставления не связанных
        с этими данными функций и по восстановлению корректности данных. Если это невозможно,
        то допускается аварийное завершение бота.
\end{enumerate}

При разработке программного кода разработчиками должны быть применены методы безопасного
программирования, которые включают:
\begin{enumerate}
    \item
        Ручную проверку кода на предмет недекларируемых возможностей
    \item
        Автоматизированную проверку кода на предмет недекларируемых возможностей
    \item
        Использование при разработке доверенной аппаратной платформы с функциями
        защиты от недекларированных возможностей на системном и прикладном уровне
    \item
        Тестирование бота
\end{enumerate}

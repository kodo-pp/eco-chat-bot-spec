\subsection{Требования к функциональности}
\label{sec:req:fn}
Чат-бот должен поддерживать следующие функции:
\begin{enumerate}
    \item
        Доступ пользователей к внутренней базе знаний
    \item
        Редактирование базы знаний пользователями, наделёнными на это правом
    \item
        Рассылка новостей, связанных с экологической повесткой, подписавшимся на рассылку
        пользователям
    \item
        Автоматический сбор новостей с интернет-ресурсов для их дальнейшей рассылки
\end{enumerate}

\subsubsection{База знаний}
    \label{sec:req:fn:kb}
    Внутренняя база знаний представляет из себя иерархическую систему из разделов
    и заметок. Вне зависимости от наполнения базы знаний информацией, в ней существует единственный
    корневой раздел, не имеющий имени. Каждый раздел, не являющийся корневым, а также
    каждая заметка принадлежат тому или иному разделу (в том числе корневому). Циклическая
    принадлежность разделов (например, раздел \(A\) принадлежит разделу \(B\), который, в свою
    очередь, принадлежит разделу \(A\)) запрещена; также никакой раздел не может принадлежать самому
    себе. Корневой раздел не принадлежит никакому разделу. Таким образом, разделы, заметки и отношение
    принадлежности образуют иерархическую структуру --- дерево разделов. Эта структура напоминает
    дерево файлов и папок в файловой системе. Схематичное изображение примера дерева разделов
    показано на рис.~\ref{fig:req:fn:kb:tree}.
    \begin{figure}[h]
        \centering
        \begingroup
        \newlength{\treeindent}
        \newlength{\treeskip}
        \setlength{\treeindent}{2em}
        \setlength{\treeskip}{-3ex}
        \newcounter{treeline}
        \setcounter{treeline}0
        \newcommand{\mypoint}[2]{(#2 * \treeindent, #1 * \treeskip)}
        \newcommand{\outpoint}[2]{(#2 * \treeindent + 0.75em, #1 * \treeskip - 1.5ex)}
        \newcommand{\midpoint}[2]{(#2 * \treeindent + 0.75em, #1 * \treeskip)}
        \newcommand{\inpoint}[2]{(#2 * \treeindent - 0.1em, #1 * \treeskip)}
        \newcommand{\mynode}[2]{
            \node at \mypoint{\thetreeline}{#1} [anchor = west] {#2};%
            \stepcounter{treeline}%
        }
        \begin{tikzpicture}
            \mynode 0 {\emjfolder{} Корневой раздел}
                \mynode 1 {\emjfolder{} Раздел <<Экология в Вышке>>}
                    \mynode 2 {\emjfolder{} Раздел <<Сбор отходов>>}
                        \mynode 3 {\emjnote{} Заметка <<Раздельный сбор мусора>>}
                        \mynode 3 {\emjnote{} Заметка <<Места сбора батареек>>}
                    \mynode 2 {\emjnote{} Заметка <<Планы развития>>}
                \mynode 1 {\emjfolder{} Раздел <<О боте>>}
                    \mynode 2 {\emjnote{} Заметка <<Справка>>}
                \mynode 1 {\emjnote{} Заметка <<Основная информация>>}
            \draw \outpoint00 -- \midpoint80;
                \draw \midpoint10 -- \inpoint11;
                \draw \midpoint60 -- \inpoint61;
                \draw \midpoint80 -- \inpoint81;
            \draw \outpoint11 -- \midpoint51;
                \draw \midpoint21 -- \inpoint22;
                \draw \midpoint51 -- \inpoint52;
            \draw \outpoint22 -- \midpoint42;
                \draw \midpoint32 -- \inpoint33;
                \draw \midpoint42 -- \inpoint43;
            \draw \outpoint61 -- \midpoint71;
                \draw \midpoint71 -- \inpoint72;
        \end{tikzpicture}
        \endgroup
        \caption{Пример дерева разделов.}
        \label{fig:req:fn:kb:tree}
    \end{figure}

    Каждый не-корневой раздел имеет имя. Каждая заметка имеет имя и содержимое.
    Именем явлется строка символов Юникода, поддерживаемых мессенджером, длиной
    не более 50~символов. У разделов и заметок, непосредственно принадлежащих одному и тому же
    разделу, имена должны различаться. Таким образом, запрещено наличие в одном разделе
    (а) двух разделов, (б) двух заметок и (в) раздела и заметки с одинаковыми именами.

    Содержимое каждой заметки является текстом длиной не более 3500~символов, а также может содержать
    не более 10~вложений. Каждое вложение --- это либо изображение разрешением не больше
    \(1920 \times 1080\)~пикселей, либо видео размером не более 50~МБ\footnote{
        Здесь и далее подразумеваются двоичные единицы измерения: \(1~\text{МБ} = 1024~\text{КБ}
        = 1048576~\text{Б}\).
    }, либо файл размером не более 50~МБ. Текст и вложения должны быть доступны пользователю
    через стандартные средства мессенджера. Допускается отправка текста и вложений
    отдельными соседними сообщениями.

    Навигация и просмотр базы знаний доступны всем пользователям.
    Пользователи, имеющие на это право, могут также редактировать базу знаний.
    Способ навигации по базе знаний, просмотра заметок и редактирования базы знаний
    описан в разделе \ref{sec:req:ui}.
    Право на редактирование описано в разделе
    \ref{sec:req:sec:privs}.

    Определённые разделы базы знаний могут быть виртуальными. Это означает, что их содержимое
    (в том числе содержимое вложенных в него разделов)
    генерируется автоматически, его нельзя создать, изменить или удалить вручную,
    а сам раздел нельзя удалить, переименовать или переместить.

    База знаний содержит один виртуальный раздел под названием <<Архив рассылок>>,
    находящийся в корневом разделе. В нём для каждой поддерживаемой рассылки
    (см.~раздел \ref{sec:req:fn:newsletter}) находится
    подраздел, в котором как заметки представлены все сообщения, которые бот присылал пользователям
    в рамках этой рассылки. Внутри раздела каждой рассылки допустимо использовать любую
    удобную структуру подразделов, однако она должна строиться по одинаковым принципам для всех рассылок.
    Разделы с рассылками, недоступными для получения определённому пользователю не должны быть ему
    видны или доступны, однако это не распространяется на рассылки, к которым пользователь имеет
    доступ, но на которые он не подписан.

    Не более 5 материалов (разделов или заметок) во всей базе знаний могут быть закреплены в
    главном меню одновременно.  Пользователи, имеющие право на редактирование базы знаний,
    могут закрепить какой-либо материал в главном меню при условии непревышения лимита на
    количество закреплённых материалов, а также открепить один или несколько материалов из
    главного меню. Порядок сортировки закреплённых материалов в главном меню определяется
    реализацией.

\subsubsection{Рассылки}
    \label{sec:req:fn:newsletter}
    Бот должен производить рассылки информационных сообщений
    пользователям, подписанным на них. Список рассылок,
    их содержимое и частота отправки указана в последующих разделах документа.
    Если для какой-либо рассылки не указана частота её отправки, отправка сообщений происходит
    незамедлительно при появлении данных. Однако в таком случае задержка в отправке сообщений допускается,
    если она обусловлена технической необходимостью или правилами, описанными ниже в данном разделе.

    В рамках организации рассылок бот не должен присылать сообщения пользователям,
    которые в данный момент находятся в интерактивном состоянии (см.~раздел~\ref{sec:req:ui:states}),
    однако он может уведомить их о том, что те или иные рассылки для них готовы.
    После выхода пользователя из интерактивного состояния бот должен повторить попытку
    рассылки не ранее, чем через 30~секунд, но не позднее, чем через 30~минут.
    Если, руководствуясь данными правилами, бот не может отправить пользователю рассылку
    более 24~часов, то допускается не отправлять данную рассылку этому пользователю в этот раз.

    Также бот не должен присылать какие-либо сообщения пользователям в первоначальном
    состоянии (см.~раздел~\ref{sec:req:ui:states}) кроме непосредственного ответа на действия
    пользователя. Отправка рассылок пользователям в этом состоянии не осуществляется. Если
    во время рассылки пользователь находится в этом состоянии, то данную рассылку он не получает.

\subsubsection{Сбор новостей}
    \label{sec:req:fn:grabnews}
    Не реже, чем один раз в день, бот совершает запросы на Интернет-ресурсы с целью получения с них
    новостей, связанных с экологической повесткой. Перечень собираемой информации:
    \begin{enumerate}
        \item
            Загловок новости
        \item
            Текст новости
        \item
            Прямая ссылка на источник новости (ресурс, с которого новость получена)
        \item
            Автор новости (если указан)
        \item
            Дата публикации (если указана)
    \end{enumerate}

    Данная операция происходит автоматически, без участия человека. При невозможности сбора данных
    с одного или нескольких ресурсов по той или иной причине бот посылает сервисное уведомление
    с описанием ошибки, а также записывает информацию об ошибке в системный журнал.

    При риске возникновения препятствий к автоматическому сбору новостей (например, CAPTCHA-проверок)
    или если правила пользования ресурса, с которого собирается информация, запрещает автоматизированный
    доступ к новостям, допускается брать новости из RSS-ленты соответствующего ресурса при её наличии.
    В таком случае, допускается не получать полный текст новости, сохраняя лишь заголовок,
    ссылку на источник и иную информацию при её наличии.

    Полученные новости сохраняются во внутреннюю базу данных бота и потом используются для новостной
    рассылки, описанной в разделе \ref{sec:req:fn:news}.

    \todo{Перечень ресурсов, с которых необходимо собирать новости.}

\subsubsection{Новостная рассылка}
    \label{sec:req:fn:news}
    Один раз в сутки бот
    должен рассылать всем пользователям, подписанным на новостную рассылку, новости с внешних
    ресурсов, которые были собраны за этот день. Пользователю
    должна выводиться вся информация, сохранённая в базе данных, перечисленная в разделе
    \ref{sec:req:fn:grabnews}. Если текст новости длиннее 3500~символов или не помещается в
    стандартное сообщение мессенджера, допускается приводить только его начало, указывая,
    что по ссылке на источник откроется полный текст новости.

    Каждая новость присылается одним сообщением, не содержит вложений и
    должна быть маркирована хэштегом \hbox{\texttt{\#новость}}.

\subsubsection{Рассылка сервисных уведомлений}
    \label{sec:req:fn:service}
    При возникновении определённых ситуаций, описанных в данном техническом задании,
    бот рассылает сервисные уведомления тем пользователям, которые на них подписаны и имеют
    право их получать. Сервисное уведомление представляет из себя одно сообщение с текстом
    не длиннее 3500~символов и не более, чем одним вложением.

    Каждое сервисное уведомление должно быть промаркировано хэштегом \hbox{\texttt{\#сервисное}}.

\subsubsection{Обратная связь}
    \label{sec:req:fn:feedback}
    Любой пользователь имеет возможность оставить свои пожелания или замечания в специальном
    разделе бота. При отправке обратной связи необходимо выбрать тему (категорию) из списка,
    представленного ниже, и написать текстовое сообщение с обратной связью.
    Также допустимо приложить не более 10 вложений (согласно определению, данному в
    разделе~\ref{sec:req:fn:kb}). Полученные сообщения рассылаются в рамках рассылки,
    описанной в разделе~\ref{sec:req:fn:feedbacknl}.

    Перечень тем обратной связи:
    \begin{itemize}
        \item
            Зелёная вышка
        \item
            Бот
        \item
            Предложить экологическую инициативу
        \item
            Другое
    \end{itemize}

\subsubsection{Рассылка обратной связи}
    \label{sec:req:fn:feedbacknl}
    При получении обратной связи от пользователей бот рассылает сообщения с ней пользователям,
    которые имеют право получать данную информацию и подписаны на данную рассылку.
    Каждое сообщение с обратной связью должно быть промаркировано хэштегом \hbox{\texttt{\#обратнаясвязь}},
    а также хэштегом, соответствующим теме обратной связи.

\subsubsection{Рассылка объявлений}
    \label{sec:req:fn:announce}
    Пользователи, имеющие право на рассылку объявлений, могут
    рассылать объявления о работе бота или Зелёной вышки всем подписанным на данную рассылку
    пользователям. Объявление предстваляет из себя одно сообщение с текстом не длиннее~3500 символов
    и не более, чем одним вложением.

    Каждое сообщение с рассылкой уведомлений пользователям должно быть промаркировано ботом хэштегом
    \hbox{\texttt{\#объявление}}

\subsubsection{Календарь событий}
    \label{sec:req:fn:calendar}
    Бот позволяет пользователям, имеющим право на управление событиями, добавлять события в календарь,
    редактировать и удалять их. Пользователям, подписанным на рассылку уведомлений о событиях,
    будут приходить уведомления при создании событий, за неделю и за день до них (допускается отправлять
    рассылку в любое время указанного дня до 20:00, кроме уведомлений о создании события).

    Рассылка соответствующих уведомлений должна быть промаркирована хэштегом \hbox{\texttt{\#календарь}}.

\subsubsection{Управление подпиской на рассылки}
    \label{sec:req:fn:subscriptions}
    Любой пользователь имеет возможность подписаться, отписаться и проверить статус подписки на
    любые рассылки, доступные этому пользователю. Перечень рассылок, поддерживаемых ботом:
    \begin{enumerate}
        \item
            Новостная рассылка, описанная в разделе \ref{sec:req:fn:news}.
            Доступна всем пользователям, по умолчанию подписка неактивна.
        \item
            Рассылка сервисных уведомлений, описанная в разделе \ref{sec:req:fn:service}.
            Доступна только пользователям, наделённых правом получения сервисных уведомлений,
            подписка автоматически становится активна, когда пользователя
            наделяют необходимыми правами.
        \item
            Рассылка обратной связи, описанная в разделе \ref{sec:req:fn:feedbacknl}.
            Доступна только пользователям, наделённых правом получения обратной связи,
            подписка автоматически становится активна, когда пользователя
            наделяют необходимыми правами.
        \item
            Рассылка объявлений, описанная в разделе \ref{sec:req:fn:announce}.
            Доступна всем пользователям, по умолчанию подписка активна.
        \item
            Рассылка уведомлений о событиях, описанная в разделе \ref{sec:req:fn:calendar}.
            Доступна всем пользователям, по умолчанию подписка неактивна.
    \end{enumerate}

\subsubsection{Управление правами пользователей}
    \label{seq:req:fn:roles}
    Пользователи, наделённые правами администратора, могут выдавать другим
    пользователям права и отзывать их. Также они могут редактировать свои права.
    Поддерживаемые права пользователей описаны в разделе \ref{sec:req:sec:privs}.
    После любого редактирования прав пользователей должен остаться хотя бы один
    пользователь с правами администратора. Попытка изменения прав, не соблюдающая
    это ограничение, заканчивается ошибкой и отсутствием изменений в правах.

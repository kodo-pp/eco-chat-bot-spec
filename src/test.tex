\section{Порядок контроля и приёмки}
\label{sec:test}
На дату ввода в эксплуатацию бот должен быть учтён Разработчиком.
Документация на бот должна содержать информацию о составе,
предназначении бота на всех этапах его жизненного цикла и
об условиях его создания и технической поддержки.

До введения в эксплуатацию бот должен пройти тестирование в соответствии с данным документом.

Порядок перевода бота из тестовой в реальную эксплуатацию осуществляется в соответствии
с календарным планом работ, определённых в договоре на его введение и при наличии
утверждённого технического задания и отчёта о тестовой эксплуатации бота.
Акт о готовности бота к вводу в реальную эксплуатацию подписывается Заказчиком.

Бот должен пройти следующие виды испытаний:
\begin{enumerate}
    \item
        Тестирование на прерывание связи с пользователем
    \item
        Тестирование на прерывание связи с интегрируемыми ресурсами
    \item
        Тестирование на долгосрочную работу
\end{enumerate}

При приёмке не допускается продолжение использования существующего секретного токена или пароля
(см. раздел~\ref{sec:req:sec:token}). Заказчик должен самостоятельно получить или
сгенерировать секретный токен, а Разработчик обязан предоставить ему инструкцию по
получению или генерации такого токена. Ответственность за корректность такой инструкции
несёт Разработчик. Ответственность за сохранение секретности сгенерированного или полученного им
токена несёт Заказчик, если программа соблюдает все перечисленные в настоящем документе требования
безопасности и надёжности.
